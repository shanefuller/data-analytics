\documentclass[]{article}
\usepackage{lmodern}
\usepackage{amssymb,amsmath}
\usepackage{ifxetex,ifluatex}
\usepackage{fixltx2e} % provides \textsubscript
\ifnum 0\ifxetex 1\fi\ifluatex 1\fi=0 % if pdftex
  \usepackage[T1]{fontenc}
  \usepackage[utf8]{inputenc}
\else % if luatex or xelatex
  \ifxetex
    \usepackage{mathspec}
  \else
    \usepackage{fontspec}
  \fi
  \defaultfontfeatures{Ligatures=TeX,Scale=MatchLowercase}
\fi
% use upquote if available, for straight quotes in verbatim environments
\IfFileExists{upquote.sty}{\usepackage{upquote}}{}
% use microtype if available
\IfFileExists{microtype.sty}{%
\usepackage{microtype}
\UseMicrotypeSet[protrusion]{basicmath} % disable protrusion for tt fonts
}{}
\usepackage[margin=1in]{geometry}
\usepackage{hyperref}
\hypersetup{unicode=true,
            pdftitle={HW 9: Soccer stats},
            pdfauthor={Shane Fuller},
            pdfborder={0 0 0},
            breaklinks=true}
\urlstyle{same}  % don't use monospace font for urls
\usepackage{color}
\usepackage{fancyvrb}
\newcommand{\VerbBar}{|}
\newcommand{\VERB}{\Verb[commandchars=\\\{\}]}
\DefineVerbatimEnvironment{Highlighting}{Verbatim}{commandchars=\\\{\}}
% Add ',fontsize=\small' for more characters per line
\usepackage{framed}
\definecolor{shadecolor}{RGB}{248,248,248}
\newenvironment{Shaded}{\begin{snugshade}}{\end{snugshade}}
\newcommand{\AlertTok}[1]{\textcolor[rgb]{0.94,0.16,0.16}{#1}}
\newcommand{\AnnotationTok}[1]{\textcolor[rgb]{0.56,0.35,0.01}{\textbf{\textit{#1}}}}
\newcommand{\AttributeTok}[1]{\textcolor[rgb]{0.77,0.63,0.00}{#1}}
\newcommand{\BaseNTok}[1]{\textcolor[rgb]{0.00,0.00,0.81}{#1}}
\newcommand{\BuiltInTok}[1]{#1}
\newcommand{\CharTok}[1]{\textcolor[rgb]{0.31,0.60,0.02}{#1}}
\newcommand{\CommentTok}[1]{\textcolor[rgb]{0.56,0.35,0.01}{\textit{#1}}}
\newcommand{\CommentVarTok}[1]{\textcolor[rgb]{0.56,0.35,0.01}{\textbf{\textit{#1}}}}
\newcommand{\ConstantTok}[1]{\textcolor[rgb]{0.00,0.00,0.00}{#1}}
\newcommand{\ControlFlowTok}[1]{\textcolor[rgb]{0.13,0.29,0.53}{\textbf{#1}}}
\newcommand{\DataTypeTok}[1]{\textcolor[rgb]{0.13,0.29,0.53}{#1}}
\newcommand{\DecValTok}[1]{\textcolor[rgb]{0.00,0.00,0.81}{#1}}
\newcommand{\DocumentationTok}[1]{\textcolor[rgb]{0.56,0.35,0.01}{\textbf{\textit{#1}}}}
\newcommand{\ErrorTok}[1]{\textcolor[rgb]{0.64,0.00,0.00}{\textbf{#1}}}
\newcommand{\ExtensionTok}[1]{#1}
\newcommand{\FloatTok}[1]{\textcolor[rgb]{0.00,0.00,0.81}{#1}}
\newcommand{\FunctionTok}[1]{\textcolor[rgb]{0.00,0.00,0.00}{#1}}
\newcommand{\ImportTok}[1]{#1}
\newcommand{\InformationTok}[1]{\textcolor[rgb]{0.56,0.35,0.01}{\textbf{\textit{#1}}}}
\newcommand{\KeywordTok}[1]{\textcolor[rgb]{0.13,0.29,0.53}{\textbf{#1}}}
\newcommand{\NormalTok}[1]{#1}
\newcommand{\OperatorTok}[1]{\textcolor[rgb]{0.81,0.36,0.00}{\textbf{#1}}}
\newcommand{\OtherTok}[1]{\textcolor[rgb]{0.56,0.35,0.01}{#1}}
\newcommand{\PreprocessorTok}[1]{\textcolor[rgb]{0.56,0.35,0.01}{\textit{#1}}}
\newcommand{\RegionMarkerTok}[1]{#1}
\newcommand{\SpecialCharTok}[1]{\textcolor[rgb]{0.00,0.00,0.00}{#1}}
\newcommand{\SpecialStringTok}[1]{\textcolor[rgb]{0.31,0.60,0.02}{#1}}
\newcommand{\StringTok}[1]{\textcolor[rgb]{0.31,0.60,0.02}{#1}}
\newcommand{\VariableTok}[1]{\textcolor[rgb]{0.00,0.00,0.00}{#1}}
\newcommand{\VerbatimStringTok}[1]{\textcolor[rgb]{0.31,0.60,0.02}{#1}}
\newcommand{\WarningTok}[1]{\textcolor[rgb]{0.56,0.35,0.01}{\textbf{\textit{#1}}}}
\usepackage{graphicx,grffile}
\makeatletter
\def\maxwidth{\ifdim\Gin@nat@width>\linewidth\linewidth\else\Gin@nat@width\fi}
\def\maxheight{\ifdim\Gin@nat@height>\textheight\textheight\else\Gin@nat@height\fi}
\makeatother
% Scale images if necessary, so that they will not overflow the page
% margins by default, and it is still possible to overwrite the defaults
% using explicit options in \includegraphics[width, height, ...]{}
\setkeys{Gin}{width=\maxwidth,height=\maxheight,keepaspectratio}
\IfFileExists{parskip.sty}{%
\usepackage{parskip}
}{% else
\setlength{\parindent}{0pt}
\setlength{\parskip}{6pt plus 2pt minus 1pt}
}
\setlength{\emergencystretch}{3em}  % prevent overfull lines
\providecommand{\tightlist}{%
  \setlength{\itemsep}{0pt}\setlength{\parskip}{0pt}}
\setcounter{secnumdepth}{0}
% Redefines (sub)paragraphs to behave more like sections
\ifx\paragraph\undefined\else
\let\oldparagraph\paragraph
\renewcommand{\paragraph}[1]{\oldparagraph{#1}\mbox{}}
\fi
\ifx\subparagraph\undefined\else
\let\oldsubparagraph\subparagraph
\renewcommand{\subparagraph}[1]{\oldsubparagraph{#1}\mbox{}}
\fi

%%% Use protect on footnotes to avoid problems with footnotes in titles
\let\rmarkdownfootnote\footnote%
\def\footnote{\protect\rmarkdownfootnote}

%%% Change title format to be more compact
\usepackage{titling}

% Create subtitle command for use in maketitle
\providecommand{\subtitle}[1]{
  \posttitle{
    \begin{center}\large#1\end{center}
    }
}

\setlength{\droptitle}{-2em}

  \title{HW 9: Soccer stats}
    \pretitle{\vspace{\droptitle}\centering\huge}
  \posttitle{\par}
    \author{Shane Fuller}
    \preauthor{\centering\large\emph}
  \postauthor{\par}
      \predate{\centering\large\emph}
  \postdate{\par}
    \date{Fall 2019}


\begin{document}
\maketitle

\hypertarget{overview}{%
\subsection{Overview}\label{overview}}

In this HW, we'll look at women's world cup data (note -- several
questions are similar to the most recent lab).

\begin{Shaded}
\begin{Highlighting}[]
\KeywordTok{library}\NormalTok{(RCurl)}
\KeywordTok{library}\NormalTok{(tidyverse)}
\NormalTok{url <-}\StringTok{ }\KeywordTok{getURL}\NormalTok{(}\StringTok{"https://raw.githubusercontent.com/statsbylopez/StatsSports/master/Data/sb_shot_data.csv"}\NormalTok{)}
\NormalTok{wwc_shot <-}\StringTok{ }\KeywordTok{read.csv}\NormalTok{(}\DataTypeTok{text =}\NormalTok{ url)}
\KeywordTok{names}\NormalTok{(wwc_shot)}
\end{Highlighting}
\end{Shaded}

\hypertarget{readings}{%
\subsection{Readings}\label{readings}}

Read xCommentary at \url{https://statsbomb.com/2016/10/xcommentary/}.

\begin{enumerate}
\def\labelenumi{\arabic{enumi}.}
\tightlist
\item
  ``We have commentary that doesn't understand how the game actually
  works.'' What specific example in soccer is he referring to?
\end{enumerate}

\emph{In the article, Knutson is referring to a shot that was taken in
Premier League by Theo Walcott from 7 yards outside the box, off a
half-volley, with his weak foot. Essentially commentators describe shots
like this as opportunities that are expected to go in, when in reality a
shot with those parameters is not expected to go in.}

\begin{enumerate}
\def\labelenumi{\arabic{enumi}.}
\setcounter{enumi}{1}
\tightlist
\item
  ``Instead we get''distance run" stats, which to my knowledge have
  never been proven as relevant to anything in football" -- Provide an
  example of another sport where new or different statistics has been
  provided, despite there being no obvious relevance to winning.
\end{enumerate}

\emph{The example that I have a relationship with is in the world of
eSports. Essentially, although there are some diligent grassroots
community members who are working hard to establish a data analytics
field for their respective fields, I have found that often some of the
data that has been gathered does not appear to have any relevance to
winning. For example, in the game that I play competitvely, one metric
that has been recorded recently is time spent on one side of the level,
which has no real significance to whether or not a player is
successful.}

\hypertarget{better-shot-maps}{%
\subsection{Better shot maps}\label{better-shot-maps}}

\texttt{ggplot()} has ample ways to enhance shot maps. Consider the
following maps

\begin{Shaded}
\begin{Highlighting}[]
\NormalTok{wwc_shot <-}\StringTok{ }\NormalTok{wwc_shot }\OperatorTok\StringTok{ }
\StringTok{  }\KeywordTok{mutate}\NormalTok{(}\DataTypeTok{is_goal =}\NormalTok{ shot.outcome.name }\OperatorTok{==}\StringTok{ "Goal"}\NormalTok{)}

\NormalTok{usa_shot <-}\StringTok{ }\NormalTok{wwc_shot }\OperatorTok\StringTok{ }
\StringTok{  }\KeywordTok{filter}\NormalTok{(possession_team.name }\OperatorTok{==}\StringTok{ "United States Women's"}\NormalTok{)}


\NormalTok{p1 <-}\StringTok{ }\KeywordTok{ggplot}\NormalTok{(usa_shot, }\KeywordTok{aes}\NormalTok{(location.x, location.y)) }\OperatorTok{+}\StringTok{ }
\StringTok{  }\KeywordTok{geom_point}\NormalTok{() }

\NormalTok{p2 <-}\StringTok{ }\KeywordTok{ggplot}\NormalTok{(usa_shot, }\KeywordTok{aes}\NormalTok{(location.x, location.y, }\DataTypeTok{colour =}\NormalTok{ is_goal)) }\OperatorTok{+}\StringTok{ }
\StringTok{  }\KeywordTok{geom_point}\NormalTok{() }

\NormalTok{p3 <-}\StringTok{ }\KeywordTok{ggplot}\NormalTok{(usa_shot, }\KeywordTok{aes}\NormalTok{(location.x, location.y, }
      \DataTypeTok{colour =}\NormalTok{ is_goal, }\DataTypeTok{size =}\NormalTok{ shot.statsbomb_xg)) }\OperatorTok{+}\StringTok{ }
\StringTok{  }\KeywordTok{geom_point}\NormalTok{()}

\NormalTok{p1}
\NormalTok{p2}
\NormalTok{p3}
\end{Highlighting}
\end{Shaded}

\begin{enumerate}
\def\labelenumi{\arabic{enumi}.}
\tightlist
\item
  What features are apparent in \texttt{p2} that aren't apparent in
  \texttt{p1}? What features are apparent in \texttt{p3} that aren't
  apparent in \texttt{p2}.
\end{enumerate}

\emph{The most helpful distinction between p1 and p2 is that p2 shows
whether or not the shot attempt was a goal, while p1 simply just shows
all of the shots taken. The most helpful distinction between p2 and p3
is that p3 shows the relative shot distribution, meaning that p3 not
only shows whether or not the shots were goals, but it also shows about
how likely it is that those shots were expected to go in.}

\begin{enumerate}
\def\labelenumi{\arabic{enumi}.}
\setcounter{enumi}{1}
\tightlist
\item
  The following contour plot creates lines where the team has shot in
  highest densities. The inside line is most \texttt{dense},
  corresponding to the center of where the team took shots. What
  features are apparent in \texttt{p4} that aren't apparent in
  \texttt{p3}? What is apparent in \texttt{p3} that isn't in
  \texttt{p4}?
\end{enumerate}

\begin{Shaded}
\begin{Highlighting}[]
\NormalTok{p4 <-}\StringTok{ }\NormalTok{usa_shot }\OperatorTok\StringTok{ }
\StringTok{  }\KeywordTok{ggplot}\NormalTok{(}\KeywordTok{aes}\NormalTok{(location.x, location.y)) }\OperatorTok{+}\StringTok{ }
\StringTok{  }\KeywordTok{stat_density_2d}\NormalTok{()}
\NormalTok{p4}
\end{Highlighting}
\end{Shaded}

\emph{The most helpful distinction in p4 is that this plot allows for
some useful binning in shot location density. This allows the user to
visualize different regions on the field and combine them, rather than
having scattered distributions. However, this plot does not provide any
information about whether or not the shot was a goal, which is a
valuable distinction that p3 offers.}

\begin{enumerate}
\def\labelenumi{\arabic{enumi}.}
\setcounter{enumi}{2}
\tightlist
\item
  Find another team
  \texttt{wwc\_shot\ \%\textgreater{}\%\ count(possession\_team.name)}
  and plot their shots. How do they compare to the USA Women's team?
\end{enumerate}

\begin{Shaded}
\begin{Highlighting}[]
\NormalTok{france_shot <-}\StringTok{ }\NormalTok{wwc_shot }\OperatorTok\StringTok{ }
\StringTok{  }\KeywordTok{filter}\NormalTok{(possession_team.name }\OperatorTok{==}\StringTok{ "France Women's"}\NormalTok{)}


\NormalTok{f1 <-}\StringTok{ }\KeywordTok{ggplot}\NormalTok{(france_shot, }\KeywordTok{aes}\NormalTok{(location.x, location.y)) }\OperatorTok{+}\StringTok{ }
\StringTok{  }\KeywordTok{geom_point}\NormalTok{() }

\NormalTok{f2 <-}\StringTok{ }\KeywordTok{ggplot}\NormalTok{(france_shot, }\KeywordTok{aes}\NormalTok{(location.x, location.y, }\DataTypeTok{colour =}\NormalTok{ is_goal)) }\OperatorTok{+}\StringTok{ }
\StringTok{  }\KeywordTok{geom_point}\NormalTok{() }

\NormalTok{f3 <-}\StringTok{ }\KeywordTok{ggplot}\NormalTok{(france_shot, }\KeywordTok{aes}\NormalTok{(location.x, location.y, }
      \DataTypeTok{colour =}\NormalTok{ is_goal, }\DataTypeTok{size =}\NormalTok{ shot.statsbomb_xg)) }\OperatorTok{+}\StringTok{ }
\StringTok{  }\KeywordTok{geom_point}\NormalTok{()}

\NormalTok{f4 <-}\StringTok{ }\NormalTok{france_shot }\OperatorTok\StringTok{ }
\StringTok{  }\KeywordTok{ggplot}\NormalTok{(}\KeywordTok{aes}\NormalTok{(location.x, location.y)) }\OperatorTok{+}\StringTok{ }
\StringTok{  }\KeywordTok{stat_density_2d}\NormalTok{()}

\NormalTok{f1}
\NormalTok{f2}
\NormalTok{f3}
\NormalTok{f4}
\end{Highlighting}
\end{Shaded}

\emph{From looking at the information provided in the first three
graphs, it just seems that the United States simply made more goals than
France, as there appears to be more blue on the US graph. Not only that,
but it also seems that the US took more shots in general. Furthermore,
looking at the last plot, it also seems that the US took better quality
shots, as the bins show a desired pattern of the bulk of shots coming
from the center and directly in front of the goal and spanning out,
while France has a rather strange distribution of shots taken.}

\hypertarget{goals-versus-expectation}{%
\subsection{Goals versus expectation}\label{goals-versus-expectation}}

\begin{enumerate}
\def\labelenumi{\arabic{enumi}.}
\setcounter{enumi}{3}
\tightlist
\item
  Let's investigate finishing ability on the USA team. Calculate the
  total number of goals scored by each player. Who actually scored the
  most goals?
\end{enumerate}

\begin{Shaded}
\begin{Highlighting}[]
\NormalTok{usa_shot }\OperatorTok
\StringTok{  }\KeywordTok{group_by}\NormalTok{(player.name) }\OperatorTok
\StringTok{  }\KeywordTok{summarise}\NormalTok{(}\DataTypeTok{n_goals =} \KeywordTok{sum}\NormalTok{(is_goal }\OperatorTok{==}\StringTok{ "TRUE"}\NormalTok{)) }\OperatorTok
\StringTok{  }\KeywordTok{arrange}\NormalTok{(}\OperatorTok{-}\NormalTok{n_goals)}
\end{Highlighting}
\end{Shaded}

\emph{Alex Morgan and Megan Rapinoe had the most goals with 6 each.}

\begin{enumerate}
\def\labelenumi{\arabic{enumi}.}
\setcounter{enumi}{4}
\tightlist
\item
  Calculate the number of expected goals scored by each player. Who was
  expected to score the most goals?
\end{enumerate}

\begin{Shaded}
\begin{Highlighting}[]
\NormalTok{usa_shot }\OperatorTok
\StringTok{  }\KeywordTok{group_by}\NormalTok{(player.name) }\OperatorTok
\StringTok{  }\KeywordTok{summarise}\NormalTok{(}\DataTypeTok{n_ex_goals =} \KeywordTok{sum}\NormalTok{(shot.statsbomb_xg)) }\OperatorTok
\StringTok{  }\KeywordTok{arrange}\NormalTok{(}\OperatorTok{-}\NormalTok{n_ex_goals)}
\end{Highlighting}
\end{Shaded}

\emph{Megan Rapinoe was expected to score the most goals with an
expected total of 3.95 goals.}

\begin{enumerate}
\def\labelenumi{\arabic{enumi}.}
\setcounter{enumi}{5}
\tightlist
\item
  The code below (using \texttt{group\_by(),\ summarise(),\ mutate()}),
  calculate the performance above/below expectation for each member of
  the USA team who took a shot. Who performed better than expectation?
  Below? What does the overall distribution say about the USA team?
\end{enumerate}

\begin{Shaded}
\begin{Highlighting}[]
\NormalTok{usa_shot }\OperatorTok
\StringTok{  }\KeywordTok{group_by}\NormalTok{(player.name) }\OperatorTok\StringTok{ }
\StringTok{  }\KeywordTok{summarise}\NormalTok{(}\DataTypeTok{xg =} \KeywordTok{sum}\NormalTok{(shot.statsbomb_xg), }
            \DataTypeTok{g =} \KeywordTok{sum}\NormalTok{(is_goal), }
            \DataTypeTok{n_shots =} \KeywordTok{n}\NormalTok{()) }\OperatorTok\StringTok{ }
\StringTok{  }\KeywordTok{mutate}\NormalTok{(}\DataTypeTok{ou =}\NormalTok{ g}\OperatorTok{-}\NormalTok{xg) }\OperatorTok\StringTok{ }
\StringTok{  }\KeywordTok{arrange}\NormalTok{(ou)}

\NormalTok{usa_shot}
\end{Highlighting}
\end{Shaded}

\emph{The list of players that performed better than expected includes
Alex Morgan, Rose Lavelle, and Megan Rapinoe. The list of players that
performed worse than expected includes Crystal Alyssia Dunn, Tobin
Heath, and Jessica McDonald. That being said, the only players who are
in the negative are players that didn't score a single goal, otherwise
every other player performed better than expected. Furthermore, the
player who performed the ``worst'' is has a -0.292, which is barely
below 0. Meanwhile, there are multiple players who performed almost a
full goal better than expected, which is especially phenominal in a
world cup. The USA Women's spread here is very impressive.}

\begin{enumerate}
\def\labelenumi{\arabic{enumi}.}
\setcounter{enumi}{6}
\tightlist
\item
  Annotate each line of code above to identify what it's doing.
\end{enumerate}

\begin{Shaded}
\begin{Highlighting}[]
\CommentTok{# usa_shot %>%                                  # Getting the data set with only usa_shots}
\CommentTok{#  group_by(player.name) %>%                    # Grouping the data set by individual player}
\CommentTok{#  summarise(xg = sum(shot.statsbomb_xg),       # Getting a sum of the expected goals}
\CommentTok{#            g = sum(is_goal),                  # Getting a sum of all successful goals}
\CommentTok{#            n_shots = n()) %>%                 # Getting a sum of all attempted shots}
\CommentTok{#  mutate(ou = g-xg) %>%                        # Creating new variable to see how they performed in relation to xg}
\CommentTok{#  arrange(ou)                                  # Ordering the data set from least goals to most goals}
\end{Highlighting}
\end{Shaded}

\hypertarget{practice-with-dplyr}{%
\subsection{Practice with dplyr}\label{practice-with-dplyr}}

\begin{enumerate}
\def\labelenumi{\arabic{enumi}.}
\setcounter{enumi}{7}
\tightlist
\item
  For each USA shooter, average the \texttt{TimeInPoss} and
  \texttt{DefendersBehindBall} when they took their shot. Filter to make
  sure you are only looking at players with at least 10 shots. What does
  this say about how players took shots?
\end{enumerate}

\begin{Shaded}
\begin{Highlighting}[]
\NormalTok{usa_shot }\OperatorTok
\StringTok{  }\KeywordTok{group_by}\NormalTok{(player.name) }\OperatorTok\StringTok{ }
\StringTok{  }\KeywordTok{summarise}\NormalTok{(}\DataTypeTok{ave_timeInPoss =}\NormalTok{ (}\KeywordTok{sum}\NormalTok{(TimeInPoss)}\OperatorTok{/}\KeywordTok{n}\NormalTok{()), }
            \DataTypeTok{ave_defendersBehindBall =}\NormalTok{ (}\KeywordTok{sum}\NormalTok{(DefendersBehindBall)}\OperatorTok{/}\KeywordTok{n}\NormalTok{()), }
            \DataTypeTok{n_shots =} \KeywordTok{n}\NormalTok{()) }\OperatorTok\StringTok{ }
\StringTok{  }\KeywordTok{filter}\NormalTok{(n_shots }\OperatorTok{>=}\StringTok{ }\DecValTok{10}\NormalTok{) }\OperatorTok
\StringTok{  }\KeywordTok{arrange}\NormalTok{(}\OperatorTok{-}\NormalTok{n_shots)}
\end{Highlighting}
\end{Shaded}

\emph{After looking at the data, I do not see any clear patterns in
average time in possession and defenders behind the ball. However, the
range of defenders behind ball is between about 3 and 5 defenders, and
the time of possession seems to have a larger range, as that goes from 9
seconds to 32 seconds. Alex Morgan and Carli Lloyd, who arguably had the
two best world cup performances, both had around 3 defenders behind ball
and around 20 seconds in possession time.}

\begin{enumerate}
\def\labelenumi{\arabic{enumi}.}
\setcounter{enumi}{8}
\tightlist
\item
  Among all players \texttt{wwc\_shot}, identify the player who finished
  with the \emph{most} and \emph{least} goals above expectation.
\end{enumerate}

\begin{Shaded}
\begin{Highlighting}[]
\NormalTok{wwc_shot }\OperatorTok
\StringTok{  }\KeywordTok{group_by}\NormalTok{(player.name) }\OperatorTok\StringTok{ }
\StringTok{  }\KeywordTok{summarise}\NormalTok{(}\DataTypeTok{xg =} \KeywordTok{sum}\NormalTok{(shot.statsbomb_xg), }
            \DataTypeTok{g =} \KeywordTok{sum}\NormalTok{(is_goal), }
            \DataTypeTok{n_shots =} \KeywordTok{n}\NormalTok{()) }\OperatorTok\StringTok{ }
\StringTok{  }\KeywordTok{mutate}\NormalTok{(}\DataTypeTok{ou =}\NormalTok{ g}\OperatorTok{-}\NormalTok{xg) }\OperatorTok\StringTok{ }
\StringTok{  }\KeywordTok{arrange}\NormalTok{(ou)}
\end{Highlighting}
\end{Shaded}

\emph{The player that finished with the most goals over expected was
Alex Morgan with 3.81 goals, while the player that finished with the
least goals in relation to expected was Nikita Parris with -1.705
goals.}

\begin{enumerate}
\def\labelenumi{\arabic{enumi}.}
\setcounter{enumi}{9}
\tightlist
\item
  Among all goalies (\texttt{player.name.GK}), identify the goalie who
  finished with the \emph{most} and \emph{least} goals allowed above
  expectation.
\end{enumerate}

\begin{Shaded}
\begin{Highlighting}[]
\NormalTok{wwc_shot }\OperatorTok
\StringTok{  }\KeywordTok{group_by}\NormalTok{(player.name.GK) }\OperatorTok\StringTok{ }
\StringTok{  }\KeywordTok{summarise}\NormalTok{(}\DataTypeTok{xg =} \KeywordTok{sum}\NormalTok{(shot.statsbomb_xg), }
            \DataTypeTok{g =} \KeywordTok{sum}\NormalTok{(is_goal), }
            \DataTypeTok{n_shots =} \KeywordTok{n}\NormalTok{()) }\OperatorTok\StringTok{ }
\StringTok{  }\KeywordTok{mutate}\NormalTok{(}\DataTypeTok{ou =}\NormalTok{ g}\OperatorTok{-}\NormalTok{xg) }\OperatorTok\StringTok{ }
\StringTok{  }\KeywordTok{arrange}\NormalTok{(ou)}
\end{Highlighting}
\end{Shaded}

\emph{The goalie that performed the best was Ingrid Hjelmseth, as she
was expected to allow 8.201 goals, and only allowed 6 goals. Therefore,
she saved 2.201 goals better than expected. The goalie that performed
the worst was Sukanya Chor Charoenying , as she was expected to allow
5.124 goals, and actually allowed 13 goals. Therefore, she saved 7.876
goals worse than expected.}

\begin{enumerate}
\def\labelenumi{\arabic{enumi}.}
\setcounter{enumi}{10}
\tightlist
\item
  Among all players, identify the player who took the most headers
  (\texttt{shot.body\_part.name\ ==\ "Head"}).
\end{enumerate}

\begin{Shaded}
\begin{Highlighting}[]
\NormalTok{wwc_shot }\OperatorTok
\StringTok{  }\KeywordTok{group_by}\NormalTok{(player.name) }\OperatorTok\StringTok{ }
\StringTok{  }\KeywordTok{summarise}\NormalTok{(}\DataTypeTok{headers =} \KeywordTok{sum}\NormalTok{(shot.body_part.name }\OperatorTok{==}\StringTok{ "Head"}\NormalTok{)) }\OperatorTok
\StringTok{  }\KeywordTok{arrange}\NormalTok{(}\OperatorTok{-}\NormalTok{headers)}

\NormalTok{wwc_shot}
\end{Highlighting}
\end{Shaded}

\emph{The player who took the most headers was Samantha Kerr, and she
attempted 8 headers throughout the tournament.}

\begin{enumerate}
\def\labelenumi{\arabic{enumi}.}
\setcounter{enumi}{11}
\tightlist
\item
  Among all players, estimate the goal rate given different
  \texttt{shot.technique.name}. Which of these tends to lead to have the
  highest chance of success?
\end{enumerate}

\begin{Shaded}
\begin{Highlighting}[]
\NormalTok{wwc_shot }\OperatorTok
\StringTok{  }\KeywordTok{group_by}\NormalTok{(shot.technique.name) }\OperatorTok\StringTok{ }
\StringTok{  }\KeywordTok{summarise}\NormalTok{(}\DataTypeTok{n_goal =} \KeywordTok{sum}\NormalTok{(is_goal), }
            \DataTypeTok{n_shots =} \KeywordTok{n}\NormalTok{(), }
            \DataTypeTok{goal_rate =}\NormalTok{ n_goal}\OperatorTok{/}\NormalTok{n_shots) }\OperatorTok
\StringTok{  }\KeywordTok{arrange}\NormalTok{(}\OperatorTok{-}\NormalTok{goal_rate)}
\end{Highlighting}
\end{Shaded}

\emph{It seems that the overhead kick and the backheel had the highest
proabability of going in for this world cup, while lobs and diving
headers were the least successful in goal conversion rate.}

\begin{enumerate}
\def\labelenumi{\arabic{enumi}.}
\setcounter{enumi}{12}
\tightlist
\item
  Compare the average shot distance by different play patterns
  (\texttt{play\_pattern.name}) -- which plays tend to lead to shots
  from longer distances? Shorter distances?
\end{enumerate}

\begin{Shaded}
\begin{Highlighting}[]
\NormalTok{wwc_shot }\OperatorTok
\StringTok{  }\KeywordTok{group_by}\NormalTok{(play_pattern.name) }\OperatorTok\StringTok{ }
\StringTok{  }\KeywordTok{summarise}\NormalTok{(}\DataTypeTok{ave_distance =}\NormalTok{ (}\KeywordTok{sum}\NormalTok{(DistToGoal)}\OperatorTok{/}\KeywordTok{n}\NormalTok{())) }\OperatorTok
\StringTok{  }\KeywordTok{arrange}\NormalTok{(}\OperatorTok{-}\NormalTok{ave_distance)}
\end{Highlighting}
\end{Shaded}

\emph{The plays tend to lead to longer shots include From Kick Off,
Regular Play, and From Goal Kick. The plays that tend to lead to shorter
shots would include Other, From Corner, and From Counter.}

\begin{enumerate}
\def\labelenumi{\arabic{enumi}.}
\setcounter{enumi}{13}
\tightlist
\item
  Make a goalie map -- that is, find a goalie, and make a shot map (most
  similar to \texttt{p3} above) that shows how the goalie fared in this
  tournament. Pick any goalie you want!
\end{enumerate}

\begin{Shaded}
\begin{Highlighting}[]
\NormalTok{tochukwu_oluehi_shot_map <-}\StringTok{ }\NormalTok{wwc_shot }\OperatorTok\StringTok{ }
\StringTok{  }\KeywordTok{filter}\NormalTok{(player.name.GK }\OperatorTok{==}\StringTok{ "Tochukwu Oluehi"}\NormalTok{)}

\NormalTok{t1 <-}\StringTok{ }\KeywordTok{ggplot}\NormalTok{(tochukwu_oluehi_shot_map, }\KeywordTok{aes}\NormalTok{(location.x, location.y, }
      \DataTypeTok{colour =}\NormalTok{ is_goal, }\DataTypeTok{size =}\NormalTok{ shot.statsbomb_xg)) }\OperatorTok{+}\StringTok{ }
\StringTok{      }\KeywordTok{geom_point}\NormalTok{()}

\NormalTok{t1}
\end{Highlighting}
\end{Shaded}

\hypertarget{exploration}{%
\subsection{Exploration}\label{exploration}}

A soccer coach wants to know the best places to shoot from. What would
you tell the coach? Create a grid across the field using the
\texttt{cut()} command (for both x and y), and then, within each
location, estimate the goal rate. Next, use \texttt{geom\_tile()} to
make a map of goal rates within each cell of the grid you created. For a
reminder on \texttt{cut()}, see our notes on Hosmer-Lemeshow, or
\texttt{?cut()}.


\end{document}
