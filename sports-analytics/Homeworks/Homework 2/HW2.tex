\documentclass[]{article}
\usepackage{lmodern}
\usepackage{amssymb,amsmath}
\usepackage{ifxetex,ifluatex}
\usepackage{fixltx2e} % provides \textsubscript
\ifnum 0\ifxetex 1\fi\ifluatex 1\fi=0 % if pdftex
  \usepackage[T1]{fontenc}
  \usepackage[utf8]{inputenc}
\else % if luatex or xelatex
  \ifxetex
    \usepackage{mathspec}
  \else
    \usepackage{fontspec}
  \fi
  \defaultfontfeatures{Ligatures=TeX,Scale=MatchLowercase}
\fi
% use upquote if available, for straight quotes in verbatim environments
\IfFileExists{upquote.sty}{\usepackage{upquote}}{}
% use microtype if available
\IfFileExists{microtype.sty}{%
\usepackage{microtype}
\UseMicrotypeSet[protrusion]{basicmath} % disable protrusion for tt fonts
}{}
\usepackage[margin=1in]{geometry}
\usepackage{hyperref}
\hypersetup{unicode=true,
            pdftitle={HW 2: Linear regression and prediction using MLB players},
            pdfauthor={Stats and sports class},
            pdfborder={0 0 0},
            breaklinks=true}
\urlstyle{same}  % don't use monospace font for urls
\usepackage{color}
\usepackage{fancyvrb}
\newcommand{\VerbBar}{|}
\newcommand{\VERB}{\Verb[commandchars=\\\{\}]}
\DefineVerbatimEnvironment{Highlighting}{Verbatim}{commandchars=\\\{\}}
% Add ',fontsize=\small' for more characters per line
\usepackage{framed}
\definecolor{shadecolor}{RGB}{248,248,248}
\newenvironment{Shaded}{\begin{snugshade}}{\end{snugshade}}
\newcommand{\AlertTok}[1]{\textcolor[rgb]{0.94,0.16,0.16}{#1}}
\newcommand{\AnnotationTok}[1]{\textcolor[rgb]{0.56,0.35,0.01}{\textbf{\textit{#1}}}}
\newcommand{\AttributeTok}[1]{\textcolor[rgb]{0.77,0.63,0.00}{#1}}
\newcommand{\BaseNTok}[1]{\textcolor[rgb]{0.00,0.00,0.81}{#1}}
\newcommand{\BuiltInTok}[1]{#1}
\newcommand{\CharTok}[1]{\textcolor[rgb]{0.31,0.60,0.02}{#1}}
\newcommand{\CommentTok}[1]{\textcolor[rgb]{0.56,0.35,0.01}{\textit{#1}}}
\newcommand{\CommentVarTok}[1]{\textcolor[rgb]{0.56,0.35,0.01}{\textbf{\textit{#1}}}}
\newcommand{\ConstantTok}[1]{\textcolor[rgb]{0.00,0.00,0.00}{#1}}
\newcommand{\ControlFlowTok}[1]{\textcolor[rgb]{0.13,0.29,0.53}{\textbf{#1}}}
\newcommand{\DataTypeTok}[1]{\textcolor[rgb]{0.13,0.29,0.53}{#1}}
\newcommand{\DecValTok}[1]{\textcolor[rgb]{0.00,0.00,0.81}{#1}}
\newcommand{\DocumentationTok}[1]{\textcolor[rgb]{0.56,0.35,0.01}{\textbf{\textit{#1}}}}
\newcommand{\ErrorTok}[1]{\textcolor[rgb]{0.64,0.00,0.00}{\textbf{#1}}}
\newcommand{\ExtensionTok}[1]{#1}
\newcommand{\FloatTok}[1]{\textcolor[rgb]{0.00,0.00,0.81}{#1}}
\newcommand{\FunctionTok}[1]{\textcolor[rgb]{0.00,0.00,0.00}{#1}}
\newcommand{\ImportTok}[1]{#1}
\newcommand{\InformationTok}[1]{\textcolor[rgb]{0.56,0.35,0.01}{\textbf{\textit{#1}}}}
\newcommand{\KeywordTok}[1]{\textcolor[rgb]{0.13,0.29,0.53}{\textbf{#1}}}
\newcommand{\NormalTok}[1]{#1}
\newcommand{\OperatorTok}[1]{\textcolor[rgb]{0.81,0.36,0.00}{\textbf{#1}}}
\newcommand{\OtherTok}[1]{\textcolor[rgb]{0.56,0.35,0.01}{#1}}
\newcommand{\PreprocessorTok}[1]{\textcolor[rgb]{0.56,0.35,0.01}{\textit{#1}}}
\newcommand{\RegionMarkerTok}[1]{#1}
\newcommand{\SpecialCharTok}[1]{\textcolor[rgb]{0.00,0.00,0.00}{#1}}
\newcommand{\SpecialStringTok}[1]{\textcolor[rgb]{0.31,0.60,0.02}{#1}}
\newcommand{\StringTok}[1]{\textcolor[rgb]{0.31,0.60,0.02}{#1}}
\newcommand{\VariableTok}[1]{\textcolor[rgb]{0.00,0.00,0.00}{#1}}
\newcommand{\VerbatimStringTok}[1]{\textcolor[rgb]{0.31,0.60,0.02}{#1}}
\newcommand{\WarningTok}[1]{\textcolor[rgb]{0.56,0.35,0.01}{\textbf{\textit{#1}}}}
\usepackage{graphicx,grffile}
\makeatletter
\def\maxwidth{\ifdim\Gin@nat@width>\linewidth\linewidth\else\Gin@nat@width\fi}
\def\maxheight{\ifdim\Gin@nat@height>\textheight\textheight\else\Gin@nat@height\fi}
\makeatother
% Scale images if necessary, so that they will not overflow the page
% margins by default, and it is still possible to overwrite the defaults
% using explicit options in \includegraphics[width, height, ...]{}
\setkeys{Gin}{width=\maxwidth,height=\maxheight,keepaspectratio}
\IfFileExists{parskip.sty}{%
\usepackage{parskip}
}{% else
\setlength{\parindent}{0pt}
\setlength{\parskip}{6pt plus 2pt minus 1pt}
}
\setlength{\emergencystretch}{3em}  % prevent overfull lines
\providecommand{\tightlist}{%
  \setlength{\itemsep}{0pt}\setlength{\parskip}{0pt}}
\setcounter{secnumdepth}{0}
% Redefines (sub)paragraphs to behave more like sections
\ifx\paragraph\undefined\else
\let\oldparagraph\paragraph
\renewcommand{\paragraph}[1]{\oldparagraph{#1}\mbox{}}
\fi
\ifx\subparagraph\undefined\else
\let\oldsubparagraph\subparagraph
\renewcommand{\subparagraph}[1]{\oldsubparagraph{#1}\mbox{}}
\fi

%%% Use protect on footnotes to avoid problems with footnotes in titles
\let\rmarkdownfootnote\footnote%
\def\footnote{\protect\rmarkdownfootnote}

%%% Change title format to be more compact
\usepackage{titling}

% Create subtitle command for use in maketitle
\providecommand{\subtitle}[1]{
  \posttitle{
    \begin{center}\large#1\end{center}
    }
}

\setlength{\droptitle}{-2em}

  \title{HW 2: Linear regression and prediction using MLB players}
    \pretitle{\vspace{\droptitle}\centering\huge}
  \posttitle{\par}
    \author{Stats and sports class}
    \preauthor{\centering\large\emph}
  \postauthor{\par}
      \predate{\centering\large\emph}
  \postdate{\par}
    \date{Fall 2019}


\begin{document}
\maketitle

\hypertarget{preliminary-notes-for-doing-hw}{%
\section{Preliminary notes for doing
HW}\label{preliminary-notes-for-doing-hw}}

\begin{enumerate}
\def\labelenumi{\arabic{enumi}.}
\item
  All files should be knit and compiled using R Markdown. Knit early and
  often! I do not recommend waiting until the end of the HW to knit.
\item
  All questions should be answered completely, and, wherever applicable,
  code should be included.
\item
  If you work with a partner or group, please write the names of your
  teammates.
\item
  Copying and pasting of code is a violation of the Skidmore honor code
\end{enumerate}

\hypertarget{homework-questions}{%
\section{Homework questions}\label{homework-questions}}

\hypertarget{part-i-linear-regression-and-player-metrics}{%
\subsection{Part I: Linear regression and player
metrics}\label{part-i-linear-regression-and-player-metrics}}

Return to the \texttt{Lahman} package in R, and we'll use the
\texttt{Batting} data frame. Type \texttt{?Batting} for specific insight
into each variable. Primarily, it's a table with 22 batting metrics.
*For all questions, we'll be using the \texttt{Batting\_1} data frame.

\begin{Shaded}
\begin{Highlighting}[]
\KeywordTok{library}\NormalTok{(tidyverse)}
\KeywordTok{library}\NormalTok{(Lahman)}
\NormalTok{Batting_}\DecValTok{1}\NormalTok{ <-}\StringTok{ }\NormalTok{Batting }\OperatorTok\StringTok{ }
\StringTok{  }\KeywordTok{filter}\NormalTok{(yearID }\OperatorTok{>=}\StringTok{ }\DecValTok{2000}\NormalTok{) }\OperatorTok\StringTok{ }
\StringTok{  }\KeywordTok{select}\NormalTok{ (playerID, yearID, AB}\OperatorTok{:}\NormalTok{SO) }\OperatorTok\StringTok{ }
\StringTok{  }\KeywordTok{filter}\NormalTok{(AB }\OperatorTok{>=}\StringTok{ }\DecValTok{500}\NormalTok{)}
\NormalTok{Batting_}\DecValTok{1}
\end{Highlighting}
\end{Shaded}

\begin{enumerate}
\def\labelenumi{\arabic{enumi}.}
\tightlist
\item
  Describe the contents of \texttt{Batting\_1}: that is, provide its
  dimensions, and what each row in the data set corresponds to.
\end{enumerate}

\emph{Batting\_1 is a dataset that contains the hitting metrics for all
batters after the year 1999 with at least 500 at bats for that season.
Each row corresponds to one player in one season, and includes relevant
batting metrics such as runs, hits, walks, and strikeouts. In this
dataset there are 2003 entries, and each row has 13 corresponding
variables.}

\begin{enumerate}
\def\labelenumi{\arabic{enumi}.}
\setcounter{enumi}{1}
\tightlist
\item
  When dealing with the \texttt{Teams} data set -- as in our labs and
  prior homework -- we often filtered by year. In the \texttt{Batting}
  data set, we are filtering by year and requiring an at-bat minimum.
  Why is this second step often required when working with players but
  not when working with teams?
\end{enumerate}

\emph{The second step is required on a player level because without
filtering on a minimum number of at bats, this could result in plenty of
extra rows that have the potential to skew the data. For example, in
baseball minor league players are often brought up to play in the
majors, and it is common that those players are not successful and they
are sent back down to the minor leagues again. If this happens muliple
times in a season for multiple teams, this could create a few hundred
entries of players with very low and unhelpful values for data
analysis.}

\begin{enumerate}
\def\labelenumi{\arabic{enumi}.}
\setcounter{enumi}{2}
\tightlist
\item
  Make a correlation matrix - both a matrix of the variables, as well as
  a visualization -- using hits, doubles, triples, home runs, RBI, and
  strikeouts.
\end{enumerate}

\begin{Shaded}
\begin{Highlighting}[]
\NormalTok{Batting_}\DecValTok{1}\NormalTok{_numeric <-}\StringTok{ }\KeywordTok{select}\NormalTok{(Batting_}\DecValTok{1}\NormalTok{, H, X2B, X3B, HR, RBI, SO)}
\NormalTok{correlation =}\StringTok{ }\KeywordTok{cor}\NormalTok{(Batting_}\DecValTok{1}\NormalTok{_numeric)}
\KeywordTok{library}\NormalTok{(corrplot)}
\KeywordTok{round}\NormalTok{(correlation, }\DecValTok{3}\NormalTok{)}
\KeywordTok{corrplot}\NormalTok{(correlation, }\DataTypeTok{method =} \StringTok{"circle"}\NormalTok{)}
\end{Highlighting}
\end{Shaded}

\begin{enumerate}
\def\labelenumi{\arabic{enumi}.}
\setcounter{enumi}{3}
\item
  Make a scatter plot of runs batted in (RBI, the x-variable) and home
  runs (HR, the x-variable). Estimate and write the regression line
  using the \texttt{lm} command. Finally, interpret the slope and
  intercept of this line.
\item
  Pete Alonso -- currently with the New York Mets -- has hit 47 home
  runs and batted in 109 runs (as of Sept 15, 2019). Given his home
  runs, what is his residual? That is, how many more or fewer runs
  batted in has he hit than we'd expect given his home runs?
\item
  Alonso seems to have fewer runs batted in than we'd expect given his
  home runs. Provide a few explanations for this is the case.
\item
  Return to your scatter plot of RBI versus HR. Use the
  \texttt{annotate} command to add in a label (Alonso's name, or a
  symbol) with where Alonso lies. Read more about \texttt{annotate}
  here: \url{https://ggplot2.tidyverse.org/reference/annotate.html}.
  Among players hitting Alonso's number of home runs, is his RBI total
  surprising?
\item
  Run the following code:
\end{enumerate}

\begin{Shaded}
\begin{Highlighting}[]
\NormalTok{Batting_}\DecValTok{1}\NormalTok{ <-}\StringTok{ }\NormalTok{Batting_}\DecValTok{1} \OperatorTok\StringTok{ }
\StringTok{  }\KeywordTok{mutate}\NormalTok{(}\DataTypeTok{K_rate =}\NormalTok{ SO}\OperatorTok{/}\NormalTok{(AB }\OperatorTok{+}\StringTok{ }\NormalTok{BB))}

\KeywordTok{ggplot}\NormalTok{(}\DataTypeTok{data =}\NormalTok{ Batting_}\DecValTok{1}\NormalTok{, }\KeywordTok{aes}\NormalTok{(}\DataTypeTok{x =}\NormalTok{ K_rate)) }\OperatorTok{+}\StringTok{ }
\StringTok{  }\KeywordTok{geom_density}\NormalTok{()}

\KeywordTok{ggplot}\NormalTok{(}\DataTypeTok{data =}\NormalTok{ Batting_}\DecValTok{1}\NormalTok{, }\KeywordTok{aes}\NormalTok{(}\DataTypeTok{x =}\NormalTok{ K_rate, }\DataTypeTok{colour =}\NormalTok{ yearID, }\DataTypeTok{group =}\NormalTok{ yearID)) }\OperatorTok{+}\StringTok{ }
\StringTok{  }\KeywordTok{geom_density}\NormalTok{()}
\end{Highlighting}
\end{Shaded}

\begin{itemize}
\tightlist
\item
  what is \texttt{K\_rate}?
\item
  Describe the distribution of \texttt{K\_rate}: e.g, what is its
  center, shape, and spread
\item
  Describe how \texttt{K\_rate} has changed over the last two decades.
  Be precise. Have the center/shape/spread changed? If so, by how much?
\end{itemize}

\hypertarget{part-ii-predictability-of-player-metrics}{%
\subsubsection{Part II: Predictability of player
metrics}\label{part-ii-predictability-of-player-metrics}}

In the above example, we looked at strikeout rate -- that is, the
percentage of time that a player strikes out.

\begin{enumerate}
\def\labelenumi{\arabic{enumi}.}
\setcounter{enumi}{8}
\tightlist
\item
  Read the article on baseball preditability here
\end{enumerate}

\url{https://blogs.fangraphs.com/basic-hitting-metric-correlation-1955-2012-2002-2012/}.

What rate metrics in baseball are most repeatable? Which metrics are
least repeatable?

Let's assess the repeatability of the metrics in \texttt{Batting\_2},
shown below:

\begin{Shaded}
\begin{Highlighting}[]
\NormalTok{Batting_}\DecValTok{2}\NormalTok{ <-}\StringTok{ }\NormalTok{Batting_}\DecValTok{1} \OperatorTok\StringTok{ }
\StringTok{  }\KeywordTok{mutate}\NormalTok{(}\DataTypeTok{HR_rate =}\NormalTok{ HR}\OperatorTok{/}\NormalTok{(AB }\OperatorTok{+}\StringTok{ }\NormalTok{BB), }
         \DataTypeTok{BB_rate =}\NormalTok{ BB}\OperatorTok{/}\NormalTok{(AB }\OperatorTok{+}\StringTok{ }\NormalTok{BB), }
         \DataTypeTok{RBI_rate =}\NormalTok{ RBI}\OperatorTok{/}\NormalTok{(AB }\OperatorTok{+}\StringTok{ }\NormalTok{BB))}

\NormalTok{Batting_}\DecValTok{2}\NormalTok{ <-}\StringTok{ }\NormalTok{Batting_}\DecValTok{2} \OperatorTok\StringTok{ }
\StringTok{  }\KeywordTok{arrange}\NormalTok{(playerID, yearID) }\OperatorTok\StringTok{ }
\StringTok{  }\KeywordTok{group_by}\NormalTok{(playerID) }\OperatorTok\StringTok{ }
\StringTok{  }\KeywordTok{mutate}\NormalTok{(}\DataTypeTok{HR_rate_next =} \KeywordTok{lead}\NormalTok{(HR_rate), }
         \DataTypeTok{K_rate_next =} \KeywordTok{lead}\NormalTok{(K_rate), }
         \DataTypeTok{BB_rate_next =} \KeywordTok{lead}\NormalTok{(BB_rate), }
         \DataTypeTok{RBI_rate_next =} \KeywordTok{lead}\NormalTok{(RBI_rate)) }\OperatorTok\StringTok{ }
\StringTok{  }\KeywordTok{ungroup}\NormalTok{() }\OperatorTok\StringTok{ }
\StringTok{  }\KeywordTok{filter}\NormalTok{(}\OperatorTok{!}\KeywordTok{is.na}\NormalTok{(HR_rate_next)) }
\end{Highlighting}
\end{Shaded}

\emph{Note:} The code drops the last year of a players' career -- there
is no future variable to look at.

\begin{enumerate}
\def\labelenumi{\arabic{enumi}.}
\setcounter{enumi}{9}
\item
  Use (i) scatter plots and (ii) correlation coefficients to assess the
  year-over-year repeatability of strikeout rate, walk rate
  (\texttt{BB\_rate}), HR rate, and RBI rate. That is, compare each
  metric in a players' curret year to the metric that he records in the
  following year. Which of these metrics is most repeatable? Which of
  these is least repeatable?
\item
  We introduced two additional ways of assessing prediction error, mean
  absolute error and mean squared error. Here's an example of how to
  code these in R.
\end{enumerate}

\begin{Shaded}
\begin{Highlighting}[]
\NormalTok{Batting_}\DecValTok{2} \OperatorTok\StringTok{ }
\StringTok{  }\KeywordTok{summarise}\NormalTok{(}\DataTypeTok{mae_k_rate =} \KeywordTok{mean}\NormalTok{(}\KeywordTok{abs}\NormalTok{(K_rate }\OperatorTok{-}\StringTok{ }\NormalTok{K_rate_next)), }
            \DataTypeTok{mse_k_rate =} \KeywordTok{mean}\NormalTok{((K_rate }\OperatorTok{-}\StringTok{ }\NormalTok{K_rate_next)}\OperatorTok{^}\DecValTok{2}\NormalTok{))}
\end{Highlighting}
\end{Shaded}

Interpret the \texttt{mae\_k\_rate} above. How does this number relate
to the scatter plot (using \texttt{K\_rate}) in Question No.~10?

\begin{enumerate}
\def\labelenumi{\arabic{enumi}.}
\setcounter{enumi}{11}
\item
  Repeat the calculations in No.~11, only using \texttt{HR\_rate}
  instead of \texttt{K\_rate}.
\item
  What does the following code show?
\end{enumerate}

\begin{Shaded}
\begin{Highlighting}[]
\KeywordTok{ggplot}\NormalTok{(}\DataTypeTok{data =}\NormalTok{ Batting_}\DecValTok{2}\NormalTok{, }\KeywordTok{aes}\NormalTok{(}\DataTypeTok{x =}\NormalTok{ yearID, }\DataTypeTok{y =}\NormalTok{ HR_rate)) }\OperatorTok{+}\StringTok{   }
\StringTok{  }\KeywordTok{geom_line}\NormalTok{(}\KeywordTok{aes}\NormalTok{(}\DataTypeTok{group =}\NormalTok{ playerID), }\DataTypeTok{colour =} \StringTok{"grey"}\NormalTok{) }\OperatorTok{+}\StringTok{ }
\StringTok{  }\KeywordTok{geom_point}\NormalTok{(}\KeywordTok{aes}\NormalTok{(}\DataTypeTok{group =}\NormalTok{ playerID), }\DataTypeTok{colour =} \StringTok{"grey"}\NormalTok{) }\OperatorTok{+}\StringTok{ }
\StringTok{  }\KeywordTok{geom_smooth}\NormalTok{()}
\end{Highlighting}
\end{Shaded}

\begin{enumerate}
\def\labelenumi{\arabic{enumi}.}
\setcounter{enumi}{13}
\tightlist
\item
  Repeat the code in No.~13, only for HR rate. Have hitters been hitting
  less home runs with the increase in strikeouts?
\end{enumerate}


\end{document}
