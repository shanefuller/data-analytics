\documentclass[]{article}
\usepackage{lmodern}
\usepackage{amssymb,amsmath}
\usepackage{ifxetex,ifluatex}
\usepackage{fixltx2e} % provides \textsubscript
\ifnum 0\ifxetex 1\fi\ifluatex 1\fi=0 % if pdftex
  \usepackage[T1]{fontenc}
  \usepackage[utf8]{inputenc}
\else % if luatex or xelatex
  \ifxetex
    \usepackage{mathspec}
  \else
    \usepackage{fontspec}
  \fi
  \defaultfontfeatures{Ligatures=TeX,Scale=MatchLowercase}
\fi
% use upquote if available, for straight quotes in verbatim environments
\IfFileExists{upquote.sty}{\usepackage{upquote}}{}
% use microtype if available
\IfFileExists{microtype.sty}{%
\usepackage{microtype}
\UseMicrotypeSet[protrusion]{basicmath} % disable protrusion for tt fonts
}{}
\usepackage[margin=1in]{geometry}
\usepackage{hyperref}
\hypersetup{unicode=true,
            pdftitle={HW 3: Multivariate regression in MLB},
            pdfauthor={Shane Fuller},
            pdfborder={0 0 0},
            breaklinks=true}
\urlstyle{same}  % don't use monospace font for urls
\usepackage{color}
\usepackage{fancyvrb}
\newcommand{\VerbBar}{|}
\newcommand{\VERB}{\Verb[commandchars=\\\{\}]}
\DefineVerbatimEnvironment{Highlighting}{Verbatim}{commandchars=\\\{\}}
% Add ',fontsize=\small' for more characters per line
\usepackage{framed}
\definecolor{shadecolor}{RGB}{248,248,248}
\newenvironment{Shaded}{\begin{snugshade}}{\end{snugshade}}
\newcommand{\AlertTok}[1]{\textcolor[rgb]{0.94,0.16,0.16}{#1}}
\newcommand{\AnnotationTok}[1]{\textcolor[rgb]{0.56,0.35,0.01}{\textbf{\textit{#1}}}}
\newcommand{\AttributeTok}[1]{\textcolor[rgb]{0.77,0.63,0.00}{#1}}
\newcommand{\BaseNTok}[1]{\textcolor[rgb]{0.00,0.00,0.81}{#1}}
\newcommand{\BuiltInTok}[1]{#1}
\newcommand{\CharTok}[1]{\textcolor[rgb]{0.31,0.60,0.02}{#1}}
\newcommand{\CommentTok}[1]{\textcolor[rgb]{0.56,0.35,0.01}{\textit{#1}}}
\newcommand{\CommentVarTok}[1]{\textcolor[rgb]{0.56,0.35,0.01}{\textbf{\textit{#1}}}}
\newcommand{\ConstantTok}[1]{\textcolor[rgb]{0.00,0.00,0.00}{#1}}
\newcommand{\ControlFlowTok}[1]{\textcolor[rgb]{0.13,0.29,0.53}{\textbf{#1}}}
\newcommand{\DataTypeTok}[1]{\textcolor[rgb]{0.13,0.29,0.53}{#1}}
\newcommand{\DecValTok}[1]{\textcolor[rgb]{0.00,0.00,0.81}{#1}}
\newcommand{\DocumentationTok}[1]{\textcolor[rgb]{0.56,0.35,0.01}{\textbf{\textit{#1}}}}
\newcommand{\ErrorTok}[1]{\textcolor[rgb]{0.64,0.00,0.00}{\textbf{#1}}}
\newcommand{\ExtensionTok}[1]{#1}
\newcommand{\FloatTok}[1]{\textcolor[rgb]{0.00,0.00,0.81}{#1}}
\newcommand{\FunctionTok}[1]{\textcolor[rgb]{0.00,0.00,0.00}{#1}}
\newcommand{\ImportTok}[1]{#1}
\newcommand{\InformationTok}[1]{\textcolor[rgb]{0.56,0.35,0.01}{\textbf{\textit{#1}}}}
\newcommand{\KeywordTok}[1]{\textcolor[rgb]{0.13,0.29,0.53}{\textbf{#1}}}
\newcommand{\NormalTok}[1]{#1}
\newcommand{\OperatorTok}[1]{\textcolor[rgb]{0.81,0.36,0.00}{\textbf{#1}}}
\newcommand{\OtherTok}[1]{\textcolor[rgb]{0.56,0.35,0.01}{#1}}
\newcommand{\PreprocessorTok}[1]{\textcolor[rgb]{0.56,0.35,0.01}{\textit{#1}}}
\newcommand{\RegionMarkerTok}[1]{#1}
\newcommand{\SpecialCharTok}[1]{\textcolor[rgb]{0.00,0.00,0.00}{#1}}
\newcommand{\SpecialStringTok}[1]{\textcolor[rgb]{0.31,0.60,0.02}{#1}}
\newcommand{\StringTok}[1]{\textcolor[rgb]{0.31,0.60,0.02}{#1}}
\newcommand{\VariableTok}[1]{\textcolor[rgb]{0.00,0.00,0.00}{#1}}
\newcommand{\VerbatimStringTok}[1]{\textcolor[rgb]{0.31,0.60,0.02}{#1}}
\newcommand{\WarningTok}[1]{\textcolor[rgb]{0.56,0.35,0.01}{\textbf{\textit{#1}}}}
\usepackage{graphicx,grffile}
\makeatletter
\def\maxwidth{\ifdim\Gin@nat@width>\linewidth\linewidth\else\Gin@nat@width\fi}
\def\maxheight{\ifdim\Gin@nat@height>\textheight\textheight\else\Gin@nat@height\fi}
\makeatother
% Scale images if necessary, so that they will not overflow the page
% margins by default, and it is still possible to overwrite the defaults
% using explicit options in \includegraphics[width, height, ...]{}
\setkeys{Gin}{width=\maxwidth,height=\maxheight,keepaspectratio}
\IfFileExists{parskip.sty}{%
\usepackage{parskip}
}{% else
\setlength{\parindent}{0pt}
\setlength{\parskip}{6pt plus 2pt minus 1pt}
}
\setlength{\emergencystretch}{3em}  % prevent overfull lines
\providecommand{\tightlist}{%
  \setlength{\itemsep}{0pt}\setlength{\parskip}{0pt}}
\setcounter{secnumdepth}{0}
% Redefines (sub)paragraphs to behave more like sections
\ifx\paragraph\undefined\else
\let\oldparagraph\paragraph
\renewcommand{\paragraph}[1]{\oldparagraph{#1}\mbox{}}
\fi
\ifx\subparagraph\undefined\else
\let\oldsubparagraph\subparagraph
\renewcommand{\subparagraph}[1]{\oldsubparagraph{#1}\mbox{}}
\fi

%%% Use protect on footnotes to avoid problems with footnotes in titles
\let\rmarkdownfootnote\footnote%
\def\footnote{\protect\rmarkdownfootnote}

%%% Change title format to be more compact
\usepackage{titling}

% Create subtitle command for use in maketitle
\providecommand{\subtitle}[1]{
  \posttitle{
    \begin{center}\large#1\end{center}
    }
}

\setlength{\droptitle}{-2em}

  \title{HW 3: Multivariate regression in MLB}
    \pretitle{\vspace{\droptitle}\centering\huge}
  \posttitle{\par}
    \author{Shane Fuller}
    \preauthor{\centering\large\emph}
  \postauthor{\par}
      \predate{\centering\large\emph}
  \postdate{\par}
    \date{Fall 2019}


\begin{document}
\maketitle

\hypertarget{preliminary-notes-for-doing-hw}{%
\section{Preliminary notes for doing
HW}\label{preliminary-notes-for-doing-hw}}

\begin{enumerate}
\def\labelenumi{\arabic{enumi}.}
\item
  All files should be knit and compiled using R Markdown. Knit early and
  often! I do not recommend waiting until the end of the HW to knit.
\item
  All questions should be answered completely, and, wherever applicable,
  code should be included.
\item
  If you work with a partner or group, please write the names of your
  teammates.
\item
  Copying and pasting of code is a violation of the Skidmore honor code
\end{enumerate}

\hypertarget{homework-questions}{%
\section{Homework questions}\label{homework-questions}}

\hypertarget{part-i-multiple-regression-and-player-metrics}{%
\subsection{Part I: Multiple regression and player
metrics}\label{part-i-multiple-regression-and-player-metrics}}

Return to the \texttt{Lahman} package in R, and we'll use the
\texttt{Teams} data frame. Below, we create a variable for the number of
singles each team had in a season.

\begin{Shaded}
\begin{Highlighting}[]
\KeywordTok{library}\NormalTok{(tidyverse)}
\KeywordTok{library}\NormalTok{(Lahman)}
\NormalTok{Teams_}\DecValTok{1}\NormalTok{ <-}\StringTok{ }\NormalTok{Teams }\OperatorTok\StringTok{ }
\StringTok{  }\KeywordTok{filter}\NormalTok{(yearID }\OperatorTok{>=}\StringTok{ }\DecValTok{2000}\NormalTok{) }\OperatorTok\StringTok{ }
\StringTok{  }\KeywordTok{mutate}\NormalTok{(}\DataTypeTok{X1B =}\NormalTok{ H }\OperatorTok{-}\StringTok{ }\NormalTok{X2B }\OperatorTok{-}\StringTok{ }\NormalTok{X3B }\OperatorTok{-}\StringTok{ }\NormalTok{HR)}
\end{Highlighting}
\end{Shaded}

\hypertarget{question-1}{%
\subsection{Question 1}\label{question-1}}

Let's use the \texttt{Teams} data set (recall: to load this data set
from the Lahman package, run the command \texttt{data(Teams)}). Using
every season since 2000, fit a multiple regression model of runs
(\texttt{R}) as a function of singles, doubles, triples, home runs, and
walks. Showing the code output is sufficient for this question.

\begin{Shaded}
\begin{Highlighting}[]
\CommentTok{#Your code goes here}
\end{Highlighting}
\end{Shaded}

\hypertarget{question-2}{%
\subsection{Question 2}\label{question-2}}

Refer to the fit in question 1. Identify the y-intercept, as well as the
slopes for singles, doubles, triples, home runs and walks (do not
interpret).

\hypertarget{question-3}{%
\subsection{Question 3}\label{question-3}}

Refer to the fit in question 1. Interpret the slope coefficient estimate
for triples.

\hypertarget{question-4}{%
\subsection{Question 4}\label{question-4}}

Use the fit in question 1 to generate a set of predicted runs scored for
each team in your data set.

What is the correlation between your predicted runs and the number of
actual runs?

\hypertarget{question-5}{%
\subsection{Question 5}\label{question-5}}

Identify the following:

\begin{enumerate}
\def\labelenumi{\roman{enumi})}
\item
  The number of runs scored by Anaheim in 2000
  (\texttt{teamID\ ==\ "ANA"})
\item
  The predicted number of runs scored by Anaheim in 2000, using your
  model in question 1.
\end{enumerate}

\hypertarget{question-6}{%
\subsection{Question 6}\label{question-6}}

Using \texttt{mutate()}, create a new variable for the residual between
the observed number of runs for each team and what your model predicted.

Next, answer:

\begin{enumerate}
\def\labelenumi{\roman{enumi})}
\item
  Which team-season corresponds to the highest residual?
\item
  Plot the residuals versus \texttt{yearID}: Is there any pattern? Would
  \texttt{yearID} be an appropriate term to add to the model?
\end{enumerate}

\hypertarget{question-7}{%
\subsection{Question 7}\label{question-7}}

Using the output from Question 1, discuss the relative importance of
each type of productive at bat (singles, doubles, triples, home runs,
walks) with respect to run generation. Does anything surprise you?

\hypertarget{question-8}{%
\subsection{Question 8}\label{question-8}}

Pick another variable in the \texttt{Teams} data set, and add it to your
regression model. Interpret it's slope. Also, does this new variable
appear to be significantly associated with runs scored, given the other
variables in the model?

\hypertarget{part-ii-model-assessment}{%
\section{Part II: Model assessment}\label{part-ii-model-assessment}}

Several models are proposed.

\begin{Shaded}
\begin{Highlighting}[]
\NormalTok{fit_}\DecValTok{1}\NormalTok{ <-}\StringTok{ }\KeywordTok{lm}\NormalTok{(R }\OperatorTok{~}\StringTok{ }\NormalTok{X1B }\OperatorTok{+}\StringTok{ }\NormalTok{X2B }\OperatorTok{+}\StringTok{ }\NormalTok{X3B }\OperatorTok{+}\StringTok{ }\NormalTok{HR, }\DataTypeTok{data =}\NormalTok{ Teams_}\DecValTok{1}\NormalTok{)}
\NormalTok{fit_}\DecValTok{2}\NormalTok{ <-}\StringTok{ }\KeywordTok{lm}\NormalTok{(R }\OperatorTok{~}\StringTok{ }\NormalTok{X1B }\OperatorTok{+}\StringTok{ }\NormalTok{X2B }\OperatorTok{+}\StringTok{ }\NormalTok{X3B }\OperatorTok{+}\StringTok{ }\NormalTok{HR }\OperatorTok{+}\StringTok{ }\NormalTok{BB, }\DataTypeTok{data =}\NormalTok{ Teams_}\DecValTok{1}\NormalTok{)}
\NormalTok{fit_}\DecValTok{3}\NormalTok{ <-}\StringTok{ }\KeywordTok{lm}\NormalTok{(R }\OperatorTok{~}\StringTok{ }\NormalTok{X1B }\OperatorTok{+}\StringTok{ }\NormalTok{X2B }\OperatorTok{+}\StringTok{ }\NormalTok{X3B }\OperatorTok{+}\StringTok{ }\NormalTok{HR }\OperatorTok{+}\StringTok{ }\NormalTok{BB }\OperatorTok{+}\StringTok{ }\NormalTok{SO, }\DataTypeTok{data =}\NormalTok{ Teams_}\DecValTok{1}\NormalTok{)}
\NormalTok{fit_}\DecValTok{4}\NormalTok{ <-}\StringTok{ }\KeywordTok{lm}\NormalTok{(R }\OperatorTok{~}\StringTok{ }\NormalTok{X1B }\OperatorTok{+}\StringTok{ }\NormalTok{X2B }\OperatorTok{+}\StringTok{ }\NormalTok{X3B }\OperatorTok{+}\StringTok{ }\NormalTok{HR }\OperatorTok{+}\StringTok{ }\NormalTok{BB }\OperatorTok{+}\StringTok{ }\NormalTok{SO }\OperatorTok{+}\StringTok{ }\NormalTok{CS, }\DataTypeTok{data =}\NormalTok{ Teams_}\DecValTok{1}\NormalTok{)}
\NormalTok{fit_}\DecValTok{5}\NormalTok{ <-}\StringTok{ }\KeywordTok{lm}\NormalTok{(R }\OperatorTok{~}\StringTok{ }\NormalTok{X1B }\OperatorTok{+}\StringTok{ }\NormalTok{X2B }\OperatorTok{+}\StringTok{ }\NormalTok{X3B }\OperatorTok{+}\StringTok{ }\NormalTok{HR }\OperatorTok{+}\StringTok{ }\NormalTok{BB }\OperatorTok{+}\StringTok{ }\NormalTok{SO }\OperatorTok{+}\StringTok{ }\NormalTok{CS }\OperatorTok{+}\StringTok{ }\NormalTok{lgID, }\DataTypeTok{data =}\NormalTok{ Teams_}\DecValTok{1}\NormalTok{)}
\NormalTok{fit_}\DecValTok{6}\NormalTok{ <-}\StringTok{ }\KeywordTok{lm}\NormalTok{(R }\OperatorTok{~}\StringTok{ }\NormalTok{X1B }\OperatorTok{+}\StringTok{ }\NormalTok{X2B }\OperatorTok{+}\StringTok{ }\NormalTok{X3B }\OperatorTok{+}\StringTok{ }\NormalTok{HR }\OperatorTok{+}\StringTok{ }\NormalTok{BB }\OperatorTok{+}\StringTok{ }\NormalTok{SO }\OperatorTok{+}\StringTok{ }\NormalTok{CS }\OperatorTok{+}\StringTok{ }\NormalTok{lgID }\OperatorTok{+}\StringTok{ }\NormalTok{SB, }\DataTypeTok{data =}\NormalTok{ Teams_}\DecValTok{1}\NormalTok{)}

\KeywordTok{options}\NormalTok{(}\DataTypeTok{scipen=}\DecValTok{999}\NormalTok{)}
\end{Highlighting}
\end{Shaded}

Note: The \texttt{options(scipen\ =\ 999)} command disables R's
scientific notation.

\hypertarget{question-9}{%
\subsection{Question 9}\label{question-9}}

Using the AIC criteria, which of the six models would you recommend for
measuring runs scored on a team-wide level? From a baseball perspective,
what does your choice suggest about certain measurements as far as their
link to runs scored?

\hypertarget{question-10}{%
\subsection{Question 10}\label{question-10}}

One of the coefficients in \texttt{fit\_5} and \texttt{fit.\_6} is
\texttt{lgID}. Generate a table of the \texttt{lgID} in your data set.
What does this variable refer to?

\hypertarget{question-11}{%
\subsection{Question 11}\label{question-11}}

Using the code below, the coefficient for \texttt{league\ =\ "NL"} is
negative. Interpret this coefficient. What about baseball's rules make
it important to consider which league each team played in? Note: you can
google the differences between the American League and the National
League to guide you.

\begin{Shaded}
\begin{Highlighting}[]
\KeywordTok{library}\NormalTok{(broom)}
\KeywordTok{tidy}\NormalTok{(fit_}\DecValTok{5}\NormalTok{)}
\end{Highlighting}
\end{Shaded}


\end{document}
