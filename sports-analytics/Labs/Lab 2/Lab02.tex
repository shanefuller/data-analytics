\documentclass[]{article}
\usepackage{lmodern}
\usepackage{amssymb,amsmath}
\usepackage{ifxetex,ifluatex}
\usepackage{fixltx2e} % provides \textsubscript
\ifnum 0\ifxetex 1\fi\ifluatex 1\fi=0 % if pdftex
  \usepackage[T1]{fontenc}
  \usepackage[utf8]{inputenc}
\else % if luatex or xelatex
  \ifxetex
    \usepackage{mathspec}
  \else
    \usepackage{fontspec}
  \fi
  \defaultfontfeatures{Ligatures=TeX,Scale=MatchLowercase}
\fi
% use upquote if available, for straight quotes in verbatim environments
\IfFileExists{upquote.sty}{\usepackage{upquote}}{}
% use microtype if available
\IfFileExists{microtype.sty}{%
\usepackage{microtype}
\UseMicrotypeSet[protrusion]{basicmath} % disable protrusion for tt fonts
}{}
\usepackage[margin=1in]{geometry}
\usepackage{hyperref}
\hypersetup{unicode=true,
            pdftitle={Correlation and regression using the Lahman database for baseball},
            pdfauthor={Michael Lopez, Skidmore College},
            pdfborder={0 0 0},
            breaklinks=true}
\urlstyle{same}  % don't use monospace font for urls
\usepackage{color}
\usepackage{fancyvrb}
\newcommand{\VerbBar}{|}
\newcommand{\VERB}{\Verb[commandchars=\\\{\}]}
\DefineVerbatimEnvironment{Highlighting}{Verbatim}{commandchars=\\\{\}}
% Add ',fontsize=\small' for more characters per line
\usepackage{framed}
\definecolor{shadecolor}{RGB}{248,248,248}
\newenvironment{Shaded}{\begin{snugshade}}{\end{snugshade}}
\newcommand{\AlertTok}[1]{\textcolor[rgb]{0.94,0.16,0.16}{#1}}
\newcommand{\AnnotationTok}[1]{\textcolor[rgb]{0.56,0.35,0.01}{\textbf{\textit{#1}}}}
\newcommand{\AttributeTok}[1]{\textcolor[rgb]{0.77,0.63,0.00}{#1}}
\newcommand{\BaseNTok}[1]{\textcolor[rgb]{0.00,0.00,0.81}{#1}}
\newcommand{\BuiltInTok}[1]{#1}
\newcommand{\CharTok}[1]{\textcolor[rgb]{0.31,0.60,0.02}{#1}}
\newcommand{\CommentTok}[1]{\textcolor[rgb]{0.56,0.35,0.01}{\textit{#1}}}
\newcommand{\CommentVarTok}[1]{\textcolor[rgb]{0.56,0.35,0.01}{\textbf{\textit{#1}}}}
\newcommand{\ConstantTok}[1]{\textcolor[rgb]{0.00,0.00,0.00}{#1}}
\newcommand{\ControlFlowTok}[1]{\textcolor[rgb]{0.13,0.29,0.53}{\textbf{#1}}}
\newcommand{\DataTypeTok}[1]{\textcolor[rgb]{0.13,0.29,0.53}{#1}}
\newcommand{\DecValTok}[1]{\textcolor[rgb]{0.00,0.00,0.81}{#1}}
\newcommand{\DocumentationTok}[1]{\textcolor[rgb]{0.56,0.35,0.01}{\textbf{\textit{#1}}}}
\newcommand{\ErrorTok}[1]{\textcolor[rgb]{0.64,0.00,0.00}{\textbf{#1}}}
\newcommand{\ExtensionTok}[1]{#1}
\newcommand{\FloatTok}[1]{\textcolor[rgb]{0.00,0.00,0.81}{#1}}
\newcommand{\FunctionTok}[1]{\textcolor[rgb]{0.00,0.00,0.00}{#1}}
\newcommand{\ImportTok}[1]{#1}
\newcommand{\InformationTok}[1]{\textcolor[rgb]{0.56,0.35,0.01}{\textbf{\textit{#1}}}}
\newcommand{\KeywordTok}[1]{\textcolor[rgb]{0.13,0.29,0.53}{\textbf{#1}}}
\newcommand{\NormalTok}[1]{#1}
\newcommand{\OperatorTok}[1]{\textcolor[rgb]{0.81,0.36,0.00}{\textbf{#1}}}
\newcommand{\OtherTok}[1]{\textcolor[rgb]{0.56,0.35,0.01}{#1}}
\newcommand{\PreprocessorTok}[1]{\textcolor[rgb]{0.56,0.35,0.01}{\textit{#1}}}
\newcommand{\RegionMarkerTok}[1]{#1}
\newcommand{\SpecialCharTok}[1]{\textcolor[rgb]{0.00,0.00,0.00}{#1}}
\newcommand{\SpecialStringTok}[1]{\textcolor[rgb]{0.31,0.60,0.02}{#1}}
\newcommand{\StringTok}[1]{\textcolor[rgb]{0.31,0.60,0.02}{#1}}
\newcommand{\VariableTok}[1]{\textcolor[rgb]{0.00,0.00,0.00}{#1}}
\newcommand{\VerbatimStringTok}[1]{\textcolor[rgb]{0.31,0.60,0.02}{#1}}
\newcommand{\WarningTok}[1]{\textcolor[rgb]{0.56,0.35,0.01}{\textbf{\textit{#1}}}}
\usepackage{graphicx,grffile}
\makeatletter
\def\maxwidth{\ifdim\Gin@nat@width>\linewidth\linewidth\else\Gin@nat@width\fi}
\def\maxheight{\ifdim\Gin@nat@height>\textheight\textheight\else\Gin@nat@height\fi}
\makeatother
% Scale images if necessary, so that they will not overflow the page
% margins by default, and it is still possible to overwrite the defaults
% using explicit options in \includegraphics[width, height, ...]{}
\setkeys{Gin}{width=\maxwidth,height=\maxheight,keepaspectratio}
\IfFileExists{parskip.sty}{%
\usepackage{parskip}
}{% else
\setlength{\parindent}{0pt}
\setlength{\parskip}{6pt plus 2pt minus 1pt}
}
\setlength{\emergencystretch}{3em}  % prevent overfull lines
\providecommand{\tightlist}{%
  \setlength{\itemsep}{0pt}\setlength{\parskip}{0pt}}
\setcounter{secnumdepth}{0}
% Redefines (sub)paragraphs to behave more like sections
\ifx\paragraph\undefined\else
\let\oldparagraph\paragraph
\renewcommand{\paragraph}[1]{\oldparagraph{#1}\mbox{}}
\fi
\ifx\subparagraph\undefined\else
\let\oldsubparagraph\subparagraph
\renewcommand{\subparagraph}[1]{\oldsubparagraph{#1}\mbox{}}
\fi

%%% Use protect on footnotes to avoid problems with footnotes in titles
\let\rmarkdownfootnote\footnote%
\def\footnote{\protect\rmarkdownfootnote}

%%% Change title format to be more compact
\usepackage{titling}

% Create subtitle command for use in maketitle
\providecommand{\subtitle}[1]{
  \posttitle{
    \begin{center}\large#1\end{center}
    }
}

\setlength{\droptitle}{-2em}

  \title{Correlation and regression using the Lahman database for baseball}
    \pretitle{\vspace{\droptitle}\centering\huge}
  \posttitle{\par}
    \author{Michael Lopez, Skidmore College}
    \preauthor{\centering\large\emph}
  \postauthor{\par}
    \date{}
    \predate{}\postdate{}
  

\begin{document}
\maketitle

\hypertarget{overview}{%
\subsection{Overview}\label{overview}}

In today's lab, we are going to explore parts of the Lahman package,
while also linking to the statistical concepts in a bivariate analysis,
including correlation, regression, and R-squared.

First, load the libraries.

\begin{Shaded}
\begin{Highlighting}[]
\KeywordTok{library}\NormalTok{(Lahman)}
\KeywordTok{library}\NormalTok{(tidyverse)}
\end{Highlighting}
\end{Shaded}

There is so much to the Lahman database, and in this course, we'll only
touch the tip of the iceberg. First, scroll to the data frame list on
the last page of the
\href{https://cran.r-project.org/web/packages/Lahman/Lahman.pdf}{tutorial}.
Under \textbf{Data sets}, there is a list of roughly 25 data sets that
come with the Lahman package.

You can also get a list of the data frames by entering the following
command.

\begin{Shaded}
\begin{Highlighting}[]
\NormalTok{LahmanData}
\end{Highlighting}
\end{Shaded}

\begin{enumerate}
\def\labelenumi{\arabic{enumi}.}
\tightlist
\item
  Which data frames in the Lahman package has the largest number of
  observations? Which have the fewest?
\end{enumerate}

We're going to start by using the \texttt{Teams} data.

\begin{Shaded}
\begin{Highlighting}[]
\KeywordTok{data}\NormalTok{(Teams)}
\KeywordTok{head}\NormalTok{(Teams)}
\KeywordTok{tail}\NormalTok{(Teams)}
\end{Highlighting}
\end{Shaded}

The \texttt{Teams} data contains team-level information for every year
of baseball season, from 1871 - 2014. That's a lot of years! To make
things a bit easier, we are going to focus on the modern era of
baseball, which is generally considered to be the period from 1970
onwards.

Let's look at teams in the modern era, using the \texttt{filter()}
command.

\begin{Shaded}
\begin{Highlighting}[]
\NormalTok{Teams_}\DecValTok{1}\NormalTok{ <-}\StringTok{ }\KeywordTok{filter}\NormalTok{(Teams, yearID }\OperatorTok{>=}\StringTok{ }\DecValTok{1970}\NormalTok{)}
\KeywordTok{head}\NormalTok{(Teams_}\DecValTok{1}\NormalTok{)}
\end{Highlighting}
\end{Shaded}

We store the newer data set as \texttt{Teams\_1}, and you can tell
\texttt{Teams\_1} is a smaller data set because it begins in 1970, and
not 1871.

Next, let's create a few missing team-level variables that we are going
to need for the lab, using the \texttt{mutate()} command.

\begin{Shaded}
\begin{Highlighting}[]
\NormalTok{Teams_}\DecValTok{1}\NormalTok{ <-}\StringTok{ }\KeywordTok{mutate}\NormalTok{(Teams_}\DecValTok{1}\NormalTok{, }\DataTypeTok{X1B =}\NormalTok{ H }\OperatorTok{-}\StringTok{ }\NormalTok{X2B }\OperatorTok{-}\StringTok{ }\NormalTok{X3B }\OperatorTok{-}\StringTok{ }\NormalTok{HR, }
                  \DataTypeTok{TB =}\NormalTok{ X1B }\OperatorTok{+}\StringTok{ }\DecValTok{2}\OperatorTok{*}\NormalTok{X2B }\OperatorTok{+}\StringTok{ }\DecValTok{3}\OperatorTok{*}\NormalTok{X3B }\OperatorTok{+}\StringTok{ }\DecValTok{4}\OperatorTok{*}\NormalTok{HR, }
                  \DataTypeTok{RC =}\NormalTok{ (H }\OperatorTok{+}\StringTok{ }\NormalTok{BB)}\OperatorTok{*}\NormalTok{TB}\OperatorTok{/}\NormalTok{(AB }\OperatorTok{+}\StringTok{ }\NormalTok{BB),}
                  \DataTypeTok{RC.new =}\NormalTok{ (H }\OperatorTok{+}\StringTok{ }\NormalTok{BB }\OperatorTok{-}\StringTok{ }\NormalTok{CS)}\OperatorTok{*}\NormalTok{(TB }\OperatorTok{+}\StringTok{ }\NormalTok{(}\FloatTok{0.55}\OperatorTok{*}\NormalTok{SB))}\OperatorTok{/}\NormalTok{(AB}\OperatorTok{+}\NormalTok{BB) )}
\end{Highlighting}
\end{Shaded}

This creates four new team-level variables, \texttt{X1B} (number of
singles), \texttt{TB} (total bases), \texttt{RC} (runs created), and
\texttt{RC.new} (a newer runs created formula). The first formula for
runs created is the basic one, discussed previously in lecture, while
the second one uses a tweak for stolen bases.

Notice that in the above code, the data name (\texttt{Teams\_1})
remained unchanged. Alternatively, we could have created a new name
(say, \texttt{Teams\_2}) if we had wanted.

\hypertarget{bivariate-analysis}{%
\subsection{Bivariate analysis}\label{bivariate-analysis}}

\hypertarget{correlation-and-scatter-plots}{%
\subsubsection{Correlation and scatter
plots}\label{correlation-and-scatter-plots}}

Using \texttt{Teams\_1}, let's quantify the strength, shape, and
direction of the association between runs created and runs.

\begin{Shaded}
\begin{Highlighting}[]
\KeywordTok{ggplot}\NormalTok{(}\DataTypeTok{data =}\NormalTok{ Teams_}\DecValTok{1}\NormalTok{, }\KeywordTok{aes}\NormalTok{(RC, R)) }\OperatorTok{+}\StringTok{ }
\StringTok{  }\KeywordTok{geom_point}\NormalTok{()}

\NormalTok{Teams_}\DecValTok{1} \OperatorTok\StringTok{ }
\StringTok{  }\KeywordTok{summarise}\NormalTok{(}\DataTypeTok{cor_var =} \KeywordTok{cor}\NormalTok{(R, RC))}
\end{Highlighting}
\end{Shaded}

\begin{enumerate}
\def\labelenumi{\arabic{enumi}.}
\setcounter{enumi}{1}
\item
  Repeat the analysis above (which uses runs created) using the newer
  formula for runs created (\texttt{RC.new})
\item
  Which variable - \texttt{RC} or \texttt{RC.new} - shows a stronger
  link with runs? What does this entail about the updated stolen bases
  formula?
\end{enumerate}

Let's identify some additional ways of quantifying the association
between two variables.

As an example, let's look at the relationship between a team's home runs
and its runs.

\begin{Shaded}
\begin{Highlighting}[]
\NormalTok{Teams_}\DecValTok{1} \OperatorTok\StringTok{ }
\StringTok{  }\KeywordTok{summarise}\NormalTok{(}\DataTypeTok{cor_var =} \KeywordTok{cor}\NormalTok{(R, HR), }
            \DataTypeTok{r_sq =}\NormalTok{ cor_var}\OperatorTok{^}\DecValTok{2}\NormalTok{)}
\end{Highlighting}
\end{Shaded}

\begin{enumerate}
\def\labelenumi{\arabic{enumi}.}
\setcounter{enumi}{3}
\tightlist
\item
  Interpret the correlation coefficient and the R-squared values between
  home runs and runs.
\end{enumerate}

\hypertarget{simple-linear-regression.}{%
\subsubsection{Simple linear
regression.}\label{simple-linear-regression.}}

From introductory statistics, you'll remember the simple linear
regression (SLR) equation. Recall, here's the estimated SLR fit:

\begin{center} $\hat{y_i} = \hat{\beta_0} + \hat{\beta_1}*x_i$ \end{center}

In the model above, \(\hat{y_i}\) represents our predicted value of an
outcome variable given \(x_i\), while \(\hat{\beta_0}\) and
\(\hat{\beta_1}\) are the estimated intercepts and slopes, respectively.

In our example, we can plug in our variable names as follows:

\begin{center} $\hat{R_i} = \hat{\beta_0} + \hat{\beta_1}*HR_i$ \end{center}

\begin{enumerate}
\def\labelenumi{\arabic{enumi}.}
\setcounter{enumi}{4}
\tightlist
\item
  Make a scatter plot of \texttt{R} as a function of \texttt{HR}. Using
  the function, make educated guesses for \(\hat{\beta_0}\) and
  \(\hat{\beta_1}\).
\end{enumerate}

Fortunately, we don't have to make educated guesses. Let's look at the
fit a model of runs as a function of home runs using the R command
\texttt{lm()}.

\begin{Shaded}
\begin{Highlighting}[]
\NormalTok{fit_}\DecValTok{1}\NormalTok{ <-}\StringTok{ }\KeywordTok{lm}\NormalTok{(R }\OperatorTok{~}\StringTok{ }\NormalTok{HR, }\DataTypeTok{data =}\NormalTok{ Teams_}\DecValTok{1}\NormalTok{)}
\KeywordTok{summary}\NormalTok{(fit_}\DecValTok{1}\NormalTok{)}
\end{Highlighting}
\end{Shaded}

The \texttt{summary()} code gives the regression output, as well as
other model characteristics.

\begin{enumerate}
\def\labelenumi{\arabic{enumi}.}
\setcounter{enumi}{5}
\item
  Write the estimated regression line. Does the actual estimated
  regression line resemble your guesses from question 5)?
\item
  Interpret both the intercept and the slope (for \texttt{HR}) in the
  estimated regression equation. Is the intercept a useful term here?
\item
  Estimate the number of runs that a team with 250 home runs would
  score. Alternatively, estimate how many home runs a team hit if they
  scored 575 runs.
\item
  Is the link between home runs and runs significant? How can you tell?
\item
  Use a similar code to determine if there is there a significant linear
  association between team runs (\texttt{R}) and the number of times
  that team was caught stealing (\texttt{CS})?
\end{enumerate}

\hypertarget{analyzing-several-variables-simultaneously}{%
\subsection{Analyzing several variables
simultaneously}\label{analyzing-several-variables-simultaneously}}

We close the lab by looking at how one would analyze a subset of
variables to judge which are most strongly associated with team success.

First, we use the \texttt{select()} command to reduce our data set from
several dozen columns to only the ones we are interested in studying.

\begin{Shaded}
\begin{Highlighting}[]
\NormalTok{Teams_}\DecValTok{2}\NormalTok{ <-}\StringTok{ }\NormalTok{Teams_}\DecValTok{1} \OperatorTok\StringTok{ }
\StringTok{  }\KeywordTok{select}\NormalTok{(R, RC.new, RC, RA, HR, SO, attendance)}
\end{Highlighting}
\end{Shaded}

Next, we calculate the pairwise correlation coefficient between each
variable.

\begin{Shaded}
\begin{Highlighting}[]
\NormalTok{cor_matrix <-}\StringTok{ }\KeywordTok{cor}\NormalTok{(Teams_}\DecValTok{2}\NormalTok{)}
\NormalTok{cor_matrix}
\KeywordTok{round}\NormalTok{(cor_matrix, }\DecValTok{3}\NormalTok{)}
\end{Highlighting}
\end{Shaded}

\begin{enumerate}
\def\labelenumi{\arabic{enumi}.}
\setcounter{enumi}{10}
\tightlist
\item
  Of the variables listed, which boast the strongest and weakest
  correlations with runs scored? Sidenote: What does the
  \texttt{round()} command do?
\end{enumerate}

This is a good time to point out that, as is usually the case with
observational data like this, strong links between two variables do not
entail that one variable causes the other. While hitting home-runs will
likely cause teams to score more runs, it is not neccessarily the case
that higher attendence causes teams to score more runs.

\begin{enumerate}
\def\labelenumi{\arabic{enumi}.}
\setcounter{enumi}{11}
\tightlist
\item
  Think of one reason why attendence and runs are significantly
  correlated besides saying that attendence causes teams to score more
  runs.
\end{enumerate}

\hypertarget{plotting-correlations}{%
\subsubsection{Plotting correlations}\label{plotting-correlations}}

There are lots of fun plots to make in R. Here are a few.

\begin{Shaded}
\begin{Highlighting}[]
\KeywordTok{library}\NormalTok{(corrplot)}
\KeywordTok{corrplot}\NormalTok{(cor_matrix, }\DataTypeTok{method=}\StringTok{"number"}\NormalTok{)}
\KeywordTok{corrplot}\NormalTok{(cor_matrix, }\DataTypeTok{method=}\StringTok{"circle"}\NormalTok{, }\DataTypeTok{type =} \StringTok{"lower"}\NormalTok{)}
\end{Highlighting}
\end{Shaded}

\hypertarget{repeatability}{%
\subsection{Repeatability}\label{repeatability}}

In addition to wanting a metric to correlate with success and to reflect
individual talent, it is worth looking at how well a metric can predict
a future, unknown performance.

This requires some careful coding. Using the \texttt{dplyr} package, we
create a new data frame, \texttt{Teams\_3}, which arranges the
team-level data by franchise and year, and then calculates the number of
runs scored in the following year as the variable \texttt{next.R}.

\begin{Shaded}
\begin{Highlighting}[]
\NormalTok{Teams_}\DecValTok{3}\NormalTok{ <-}\StringTok{ }\NormalTok{Teams_}\DecValTok{1} \OperatorTok
\StringTok{  }\KeywordTok{arrange}\NormalTok{(franchID, yearID) }\OperatorTok
\StringTok{  }\KeywordTok{group_by}\NormalTok{(franchID) }\OperatorTok
\StringTok{  }\KeywordTok{mutate}\NormalTok{(}\DataTypeTok{next_R =} \KeywordTok{lead}\NormalTok{(R))}
\KeywordTok{head}\NormalTok{(Teams_}\DecValTok{3}\NormalTok{)}
\end{Highlighting}
\end{Shaded}

\begin{enumerate}
\def\labelenumi{\arabic{enumi}.}
\setcounter{enumi}{12}
\tightlist
\item
  Verify that the last column of \texttt{Teams.2} contains the future
  runs scored for each team in each row (hint: use the \texttt{select}
  command)
\end{enumerate}

To close, we look at which team-level variables most strongly correlate
with future runs scored.

\begin{Shaded}
\begin{Highlighting}[]
\NormalTok{Teams_}\DecValTok{3}\NormalTok{ <-}\StringTok{ }\NormalTok{Teams_}\DecValTok{3} \OperatorTok\StringTok{ }\KeywordTok{ungroup}\NormalTok{() }\CommentTok{### Needed to remove grouping from earlier code}
\NormalTok{Teams_}\DecValTok{4}\NormalTok{ <-}\StringTok{ }\NormalTok{Teams_}\DecValTok{3} \OperatorTok\StringTok{ }
\StringTok{  }\KeywordTok{select}\NormalTok{(next_R, R, RC, X1B, X2B, X3B, HR, SLG, OBP, OPS, attendance)}
\NormalTok{cor_matrix <-}\StringTok{ }\KeywordTok{cor}\NormalTok{(Teams_}\DecValTok{4}\NormalTok{, }\DataTypeTok{use=}\StringTok{"pairwise.complete.obs"}\NormalTok{)}
\NormalTok{cor_matrix}
\end{Highlighting}
\end{Shaded}

\begin{enumerate}
\def\labelenumi{\arabic{enumi}.}
\setcounter{enumi}{13}
\item
  Which variables appear to be the best at predicting a team's runs
  scored in the following year? Which appears to be the worst?
\item
  Related: Compare the correlation of number of singles (\texttt{X1B})
  to \texttt{next.R} and \texttt{R}, as well as number of home runs
  (\texttt{HR}) to \texttt{next.R} and \texttt{R}. Think carefully about
  how this would impact your recommendations of how to build a team.
  Which measures should be looked at most closely? Which measures appear
  to be mostly noise?
\item
  For fun: Go to the Shiny app
  \href{https://istats.shinyapps.io/guesscorr/}{here}. Take a few
  guesses, and then click
  \texttt{View\ scatterplot\ of\ correlation\ between\ your\ guesses\ and\ actual\ correlation}.
  How good are you at guessing the correlation coefficient?
\end{enumerate}


\end{document}
