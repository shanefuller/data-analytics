\documentclass[]{article}
\usepackage{lmodern}
\usepackage{amssymb,amsmath}
\usepackage{ifxetex,ifluatex}
\usepackage{fixltx2e} % provides \textsubscript
\ifnum 0\ifxetex 1\fi\ifluatex 1\fi=0 % if pdftex
  \usepackage[T1]{fontenc}
  \usepackage[utf8]{inputenc}
\else % if luatex or xelatex
  \ifxetex
    \usepackage{mathspec}
  \else
    \usepackage{fontspec}
  \fi
  \defaultfontfeatures{Ligatures=TeX,Scale=MatchLowercase}
\fi
% use upquote if available, for straight quotes in verbatim environments
\IfFileExists{upquote.sty}{\usepackage{upquote}}{}
% use microtype if available
\IfFileExists{microtype.sty}{%
\usepackage{microtype}
\UseMicrotypeSet[protrusion]{basicmath} % disable protrusion for tt fonts
}{}
\usepackage[margin=1in]{geometry}
\usepackage{hyperref}
\hypersetup{unicode=true,
            pdftitle={Multiple regression and R-squared},
            pdfauthor={Michael Lopez, Skidmore College},
            pdfborder={0 0 0},
            breaklinks=true}
\urlstyle{same}  % don't use monospace font for urls
\usepackage{color}
\usepackage{fancyvrb}
\newcommand{\VerbBar}{|}
\newcommand{\VERB}{\Verb[commandchars=\\\{\}]}
\DefineVerbatimEnvironment{Highlighting}{Verbatim}{commandchars=\\\{\}}
% Add ',fontsize=\small' for more characters per line
\usepackage{framed}
\definecolor{shadecolor}{RGB}{248,248,248}
\newenvironment{Shaded}{\begin{snugshade}}{\end{snugshade}}
\newcommand{\AlertTok}[1]{\textcolor[rgb]{0.94,0.16,0.16}{#1}}
\newcommand{\AnnotationTok}[1]{\textcolor[rgb]{0.56,0.35,0.01}{\textbf{\textit{#1}}}}
\newcommand{\AttributeTok}[1]{\textcolor[rgb]{0.77,0.63,0.00}{#1}}
\newcommand{\BaseNTok}[1]{\textcolor[rgb]{0.00,0.00,0.81}{#1}}
\newcommand{\BuiltInTok}[1]{#1}
\newcommand{\CharTok}[1]{\textcolor[rgb]{0.31,0.60,0.02}{#1}}
\newcommand{\CommentTok}[1]{\textcolor[rgb]{0.56,0.35,0.01}{\textit{#1}}}
\newcommand{\CommentVarTok}[1]{\textcolor[rgb]{0.56,0.35,0.01}{\textbf{\textit{#1}}}}
\newcommand{\ConstantTok}[1]{\textcolor[rgb]{0.00,0.00,0.00}{#1}}
\newcommand{\ControlFlowTok}[1]{\textcolor[rgb]{0.13,0.29,0.53}{\textbf{#1}}}
\newcommand{\DataTypeTok}[1]{\textcolor[rgb]{0.13,0.29,0.53}{#1}}
\newcommand{\DecValTok}[1]{\textcolor[rgb]{0.00,0.00,0.81}{#1}}
\newcommand{\DocumentationTok}[1]{\textcolor[rgb]{0.56,0.35,0.01}{\textbf{\textit{#1}}}}
\newcommand{\ErrorTok}[1]{\textcolor[rgb]{0.64,0.00,0.00}{\textbf{#1}}}
\newcommand{\ExtensionTok}[1]{#1}
\newcommand{\FloatTok}[1]{\textcolor[rgb]{0.00,0.00,0.81}{#1}}
\newcommand{\FunctionTok}[1]{\textcolor[rgb]{0.00,0.00,0.00}{#1}}
\newcommand{\ImportTok}[1]{#1}
\newcommand{\InformationTok}[1]{\textcolor[rgb]{0.56,0.35,0.01}{\textbf{\textit{#1}}}}
\newcommand{\KeywordTok}[1]{\textcolor[rgb]{0.13,0.29,0.53}{\textbf{#1}}}
\newcommand{\NormalTok}[1]{#1}
\newcommand{\OperatorTok}[1]{\textcolor[rgb]{0.81,0.36,0.00}{\textbf{#1}}}
\newcommand{\OtherTok}[1]{\textcolor[rgb]{0.56,0.35,0.01}{#1}}
\newcommand{\PreprocessorTok}[1]{\textcolor[rgb]{0.56,0.35,0.01}{\textit{#1}}}
\newcommand{\RegionMarkerTok}[1]{#1}
\newcommand{\SpecialCharTok}[1]{\textcolor[rgb]{0.00,0.00,0.00}{#1}}
\newcommand{\SpecialStringTok}[1]{\textcolor[rgb]{0.31,0.60,0.02}{#1}}
\newcommand{\StringTok}[1]{\textcolor[rgb]{0.31,0.60,0.02}{#1}}
\newcommand{\VariableTok}[1]{\textcolor[rgb]{0.00,0.00,0.00}{#1}}
\newcommand{\VerbatimStringTok}[1]{\textcolor[rgb]{0.31,0.60,0.02}{#1}}
\newcommand{\WarningTok}[1]{\textcolor[rgb]{0.56,0.35,0.01}{\textbf{\textit{#1}}}}
\usepackage{graphicx,grffile}
\makeatletter
\def\maxwidth{\ifdim\Gin@nat@width>\linewidth\linewidth\else\Gin@nat@width\fi}
\def\maxheight{\ifdim\Gin@nat@height>\textheight\textheight\else\Gin@nat@height\fi}
\makeatother
% Scale images if necessary, so that they will not overflow the page
% margins by default, and it is still possible to overwrite the defaults
% using explicit options in \includegraphics[width, height, ...]{}
\setkeys{Gin}{width=\maxwidth,height=\maxheight,keepaspectratio}
\IfFileExists{parskip.sty}{%
\usepackage{parskip}
}{% else
\setlength{\parindent}{0pt}
\setlength{\parskip}{6pt plus 2pt minus 1pt}
}
\setlength{\emergencystretch}{3em}  % prevent overfull lines
\providecommand{\tightlist}{%
  \setlength{\itemsep}{0pt}\setlength{\parskip}{0pt}}
\setcounter{secnumdepth}{0}
% Redefines (sub)paragraphs to behave more like sections
\ifx\paragraph\undefined\else
\let\oldparagraph\paragraph
\renewcommand{\paragraph}[1]{\oldparagraph{#1}\mbox{}}
\fi
\ifx\subparagraph\undefined\else
\let\oldsubparagraph\subparagraph
\renewcommand{\subparagraph}[1]{\oldsubparagraph{#1}\mbox{}}
\fi

%%% Use protect on footnotes to avoid problems with footnotes in titles
\let\rmarkdownfootnote\footnote%
\def\footnote{\protect\rmarkdownfootnote}

%%% Change title format to be more compact
\usepackage{titling}

% Create subtitle command for use in maketitle
\providecommand{\subtitle}[1]{
  \posttitle{
    \begin{center}\large#1\end{center}
    }
}

\setlength{\droptitle}{-2em}

  \title{Multiple regression and R-squared}
    \pretitle{\vspace{\droptitle}\centering\huge}
  \posttitle{\par}
    \author{Michael Lopez, Skidmore College}
    \preauthor{\centering\large\emph}
  \postauthor{\par}
    \date{}
    \predate{}\postdate{}
  

\begin{document}
\maketitle

\hypertarget{overview}{%
\subsection{Overview}\label{overview}}

In this lab, we'll learn model comparison tools using multivariate
regression. Next, we'll apply our tools to derive predictions of pitcher
performance.

First, recall that we have to load the requisite libraries that we'll
need (and we may have to install them, too. As always, once a package is
downloaded, you do not need to run the \texttt{install.packages()} code
again.

\begin{Shaded}
\begin{Highlighting}[]
\KeywordTok{library}\NormalTok{(Lahman)}
\KeywordTok{library}\NormalTok{(tidyverse)}
\end{Highlighting}
\end{Shaded}

We're going to start by using the \texttt{Teams} data.

\begin{Shaded}
\begin{Highlighting}[]
\KeywordTok{data}\NormalTok{(Teams)}
\KeywordTok{head}\NormalTok{(Teams)}
\KeywordTok{tail}\NormalTok{(Teams)}
\end{Highlighting}
\end{Shaded}

\begin{Shaded}
\begin{Highlighting}[]
\NormalTok{Teams}\FloatTok{.1}\NormalTok{ <-}\StringTok{ }\KeywordTok{filter}\NormalTok{(Teams, yearID }\OperatorTok{>=}\StringTok{ }\DecValTok{1970}\NormalTok{)}
\KeywordTok{head}\NormalTok{(Teams}\FloatTok{.1}\NormalTok{)}
\end{Highlighting}
\end{Shaded}

\hypertarget{comparing-multiple-regression-models.}{%
\subsection{Comparing multiple regression
models.}\label{comparing-multiple-regression-models.}}

There's an old saying in statistics, attributed to George Box:
\(\textit{all models are wrong, some are useful}\). In practice, we
never know if our regression model is correctly specified; e.g, that it
is really the case the \(y\), \(x_1\), \ldots{} and \(x_{p-1}\) are
linearly related. All we can do is hope\ldots{}and try a few analytical
tools.

Let's try to come up with a few models of \texttt{RA}: runs against.
First, we start with a recap of multiple regression.

\begin{Shaded}
\begin{Highlighting}[]
\KeywordTok{library}\NormalTok{(broom)}
\NormalTok{fit}\FloatTok{.1}\NormalTok{ <-}\StringTok{ }\KeywordTok{lm}\NormalTok{(RA }\OperatorTok{~}\StringTok{ }\NormalTok{HRA }\OperatorTok{+}\StringTok{ }\NormalTok{BBA }\OperatorTok{+}\StringTok{ }\NormalTok{SOA }\OperatorTok{+}\StringTok{ }\NormalTok{HA }\OperatorTok{+}\StringTok{ }\NormalTok{attendance, }\DataTypeTok{data =}\NormalTok{ Teams}\FloatTok{.1}\NormalTok{)}
\KeywordTok{tidy}\NormalTok{(fit}\FloatTok{.1}\NormalTok{)}
\end{Highlighting}
\end{Shaded}

\begin{enumerate}
\def\labelenumi{\arabic{enumi}.}
\tightlist
\item
  Write the estimated model above.
\end{enumerate}

\emph{RA-hat = -1.54 + 1.26(HRA-hat) + 3.54(BBA-hat) - 6.53(SOA-hat) +
3.99(HA-hat) - 3.53(attendance-hat)}

\begin{enumerate}
\def\labelenumi{\arabic{enumi}.}
\setcounter{enumi}{1}
\tightlist
\item
  Using the model in question (1), interpret the coefficient on
  \texttt{HRA}.
\end{enumerate}

\emph{For every 1 home run allowed, there is an expected increase in
runs allowed by 1.25, given we have BBA, SOA, HA, and attendance in the
model.}

\begin{enumerate}
\def\labelenumi{\arabic{enumi}.}
\setcounter{enumi}{2}
\tightlist
\item
  Using the model in (1), interpret the coefficient on
  \texttt{attendance}. Then come up with a better way to interpret the
  coefficient on \texttt{attendance}.
\end{enumerate}

\emph{For every 1 home run allowed, there is an expected increase in
runs allowed by 1.25, given we have BBA, SOA, HA, and attendance in the
model.}

\begin{enumerate}
\def\labelenumi{\arabic{enumi}.}
\setcounter{enumi}{3}
\tightlist
\item
  Remove \texttt{HRA} from the model and re-fit. Do your other
  coefficients change? Can you explain the difference?
\end{enumerate}

After fitting a linear regression model, it is appropriate to check
assumptions. First, we check the appropriateness of the normal
distribution for residuals.

\begin{Shaded}
\begin{Highlighting}[]
\KeywordTok{qqnorm}\NormalTok{(fit}\FloatTok{.1}\OperatorTok{$}\NormalTok{resid)}
\KeywordTok{qqline}\NormalTok{(fit}\FloatTok{.1}\OperatorTok{$}\NormalTok{residuals)}
\end{Highlighting}
\end{Shaded}

Next, we compare the residuals to the fitted values, checking for the
assumptions of independence among the residuals, as well as the constant
variance assumption.

\begin{Shaded}
\begin{Highlighting}[]
\KeywordTok{ggplot}\NormalTok{(}\DataTypeTok{data =}\NormalTok{ fit}\FloatTok{.1}\NormalTok{, }\KeywordTok{aes}\NormalTok{(}\DataTypeTok{x =}\NormalTok{ fit}\FloatTok{.1}\OperatorTok{$}\NormalTok{fitted.values, }\DataTypeTok{y =}\NormalTok{ fit}\FloatTok{.1}\OperatorTok{$}\NormalTok{residuals)) }\OperatorTok{+}\StringTok{ }
\StringTok{         }\KeywordTok{geom_point}\NormalTok{() }\OperatorTok{+}\StringTok{ }
\StringTok{  }\KeywordTok{geom_smooth}\NormalTok{()}
\end{Highlighting}
\end{Shaded}

\begin{enumerate}
\def\labelenumi{\arabic{enumi}.}
\setcounter{enumi}{4}
\tightlist
\item
  What do the residual plots suggest about the assumptions of our linear
  regression model? What about the model makes it possibly a poor fit?
\end{enumerate}

Speaking of residuals, lets take a deeper look at individual
predictions.

The 1970 Atlanta Braves allowed 185 home runs, 478 walks, struck out 960
batters, and gave up 1451 hits. Their attendance was 1078848. As it
turns out, the Braves are the first row of our data set.

\begin{Shaded}
\begin{Highlighting}[]
\NormalTok{Teams}\FloatTok{.1} \OperatorTok\StringTok{ }
\StringTok{  }\KeywordTok{slice}\NormalTok{(}\DecValTok{1}\NormalTok{)}

\NormalTok{Teams}\FloatTok{.1}\NormalTok{ <-}\StringTok{ }\NormalTok{Teams}\FloatTok{.1} \OperatorTok\StringTok{ }
\StringTok{  }\KeywordTok{mutate}\NormalTok{(}\DataTypeTok{fitted.value =} \KeywordTok{predict}\NormalTok{(fit}\FloatTok{.1}\NormalTok{))}

\NormalTok{Teams}\FloatTok{.1} \OperatorTok\StringTok{ }
\StringTok{  }\KeywordTok{slice}\NormalTok{(}\DecValTok{1}\NormalTok{)}
\end{Highlighting}
\end{Shaded}

\begin{enumerate}
\def\labelenumi{\arabic{enumi}.}
\setcounter{enumi}{5}
\tightlist
\item
  The \texttt{predict} command calculates the fitted number of runs
  using a model such as ours (our model is \texttt{fit.1}. Using the
  code above, how many runs did our model predict that Braves to have
  allowed? What is the residual for the number of runs allowed by the
  Braves? Did our model overestimate or underestimate Atlanta's
  performance?
\end{enumerate}

\hypertarget{r-squared}{%
\subsubsection{R-squared}\label{r-squared}}

There are lots of ways to measure the success of a regression model. The
most common metric is \texttt{R-squared}, which you'll recall is the
fraction of variability in the outcome which is explained by the
regression model. Larger R-squared's are, in principal, better.

You can access the R-squared via the \texttt{summary} command:

\begin{Shaded}
\begin{Highlighting}[]
\KeywordTok{summary}\NormalTok{(fit}\FloatTok{.1}\NormalTok{)}
\end{Highlighting}
\end{Shaded}

Using our model, we would interpret the R-squared as follows:

\textit{90.09\% of the variability in the number of runs allowed by a team can be explained by the linear model with HRA, BBA, SOA, HA, and attendance}.

While popular, the traditional R-squared is also flawed. Let's see how.
In the following code, we'll create two new variables in the
\texttt{Teams.1} data set, \texttt{rand1} and \texttt{rand2}, which are
random normal variables.

\begin{Shaded}
\begin{Highlighting}[]
\KeywordTok{set.seed}\NormalTok{(}\DecValTok{0}\NormalTok{)}
\NormalTok{Teams}\FloatTok{.1}\NormalTok{ <-}\StringTok{ }\KeywordTok{mutate}\NormalTok{(Teams}\FloatTok{.1}\NormalTok{, }
                  \DataTypeTok{rand1 =} \KeywordTok{rnorm}\NormalTok{(}\KeywordTok{nrow}\NormalTok{(Teams}\FloatTok{.1}\NormalTok{)),}
                  \DataTypeTok{rand2 =} \KeywordTok{rnorm}\NormalTok{(}\KeywordTok{nrow}\NormalTok{(Teams}\FloatTok{.1}\NormalTok{)))}
\KeywordTok{head}\NormalTok{(Teams}\FloatTok{.1}\NormalTok{)}
\end{Highlighting}
\end{Shaded}

Let's see what happens when we include \texttt{rand1} and \texttt{rand2}
to our regression fit.

\begin{Shaded}
\begin{Highlighting}[]
\NormalTok{fit}\FloatTok{.2}\NormalTok{ <-}\StringTok{ }\KeywordTok{lm}\NormalTok{(RA }\OperatorTok{~}\StringTok{ }\NormalTok{HRA }\OperatorTok{+}\StringTok{ }\NormalTok{BBA }\OperatorTok{+}\StringTok{ }\NormalTok{SOA }\OperatorTok{+}\StringTok{ }\NormalTok{HA }\OperatorTok{+}\StringTok{ }\NormalTok{attendance }\OperatorTok{+}\StringTok{ }\NormalTok{rand1 }\OperatorTok{+}\StringTok{ }\NormalTok{rand2, }\DataTypeTok{data =}\NormalTok{ Teams}\FloatTok{.1}\NormalTok{)}
\KeywordTok{summary}\NormalTok{(fit}\FloatTok{.2}\NormalTok{)}
\end{Highlighting}
\end{Shaded}

Even when we added random noise to the model, R-squared went up!

That's not a good thing, at least when it comes to making model
comparisons. In fact, its a property of R-squared that, no matter what
variable you add to a given model, the R-squared cannot go down. As a
result, R-squared is not useful for model comparisons, but more to gain
a sense of how much of a drop in the variability in the outcome can be
explained by the model's fit.

In place of R-squared, R also shows a formula for an adjusted R-squared,
which penalizes models for adding unneeded parameters. However, this
metric also has weaknesses.

\begin{enumerate}
\def\labelenumi{\arabic{enumi}.}
\setcounter{enumi}{6}
\item
  When \emph{would} be an appropriate time to compare R-squared's from
  two different models?
\item
  What other approaches to picking a model may be more appropriate than
  R-squared?
\end{enumerate}


\end{document}
